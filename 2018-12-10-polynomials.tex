\documentclass{article}
\usepackage[T1]{fontenc}
\usepackage[english]{babel}
\usepackage[utf8]{inputenc}
\usepackage{fancyhdr}
\usepackage{amsmath}
\usepackage{amsfonts}
\usepackage{amssymb}
\usepackage{amsthm} 
\usepackage{thmtools}
\usepackage{lipsum}
\usepackage{geometry}
\usepackage{mathtools}
\usepackage{bold-extra}
\usepackage{mathrsfs}
\usepackage{tikz}
\usepackage{tikz-cd}
\usepackage[makeroom]{cancel}
\usepackage{hanging}
\usepackage{stmaryrd}
\usepackage{enumerate}
\usepackage{color, soul}
\usepackage{fancyhdr}
\usepackage{titlesec}
\usepackage{parskip}
\usepackage{soul}
\usepackage{graphicx}
\usepackage{mathdots}

\DeclareSymbolFont{extraup}{U}{zavm}{m}{n}
\DeclareMathSymbol{\varheart}{\mathalpha}{extraup}{86}
\DeclareMathSymbol{\vardiamond}{\mathalpha}{extraup}{87}

\usepackage{fullpage}
%theorems, etc.
\theoremstyle{definition}
\newtheorem{thm}{Theorem}[section]
%\numberwithin{thm}{subsection}
\newtheorem{lem}[thm]{Lemma}
%\numberwithin{lem}{subsection}
\newtheorem{prop}[thm]{Proposition}
%\numberwithin{prop}{subsection}
\newtheorem{cor}[thm]{Corollary}
%\numberwithin{cor}{subsection}
\newtheorem{defn}[thm]{Definition}
%\numberwithin{defn}{subsection}
\newtheorem{conj}[thm]{Conjecture}
%\numberwithin{conj}{subsection}
\newtheorem{exam}[thm]{Example}
%\numberwithin{exam}{subsection}
\newtheorem{alg}[thm]{Algorithm}
%\numberwithin{alg}{subsection}
\newtheorem{hw}[thm]{Exercise}
\newtheorem{note}[thm]{Note}
%\numberwithin{note}{subsection}
\newtheorem{rem}[thm]{Remark}
%\numberwithin{rem}{subsection}

\newcommand{\dfn}{\textbf{Definition. }}


%\numberwithin{equation}{section}

\usepackage[colorlinks]{hyperref}
\usepackage[nameinlink,capitalize]{cleveref}


%\titleformat{<command>}[<shape>]{<format>}{<label>}{<sep>}{<before-code>}[<after-code>]

\setlength\parindent{0pt}


%========= Spacing and format
\newcommand{\cen}{\centerline}
\newcommand{\hang}{\hangindent=0.8cm}
\newcommand{\nhang}{\hangindent=0cm}
\newcommand{\nf}{\normalfont}
\newcommand{\fl}{\noindent}
\newcommand{\vs}{\vspace{0.7em}}
\newcommand{\vv}{\\\vs}
\newcommand{\nl}{\vspace{4cm}\\}



%========= Common Math commands
\newcommand{\ex}{\exists}
\newcommand{\nin}{\not\in}
\newcommand{\ra}{\rightarrow}
\newcommand{\Ra}{\Rightarrow}
\newcommand{\La}{\Leftarrow}
\newcommand{\oa}{\overrightarrow}
\newcommand{\lbb}{\llbracket}
\newcommand{\rbb}{\rrbracket}
\newcommand{\p}{\prime}
\newcommand{\wh}{\widehat}
\newcommand{\os}{\overset}
\newcommand{\us}{\underset}
\newcommand{\mf}{\mathfrak}
\newcommand{\ol}{\overline}
\newcommand{\td}{\widetilde}
\newcommand{\seq}{\subseteq}
\newcommand{\lp}{\left(}
\newcommand{\rp}{\right)}
\newcommand{\im}{\text{im}}
\newcommand{\inv}{^{-1}}




%========= Math letters
\newcommand{\R}{\mathbb{R}}
\newcommand{\C}{\mathbb{C}}
\newcommand{\Q}{\mathbb{Q}}
\newcommand{\Z}{\mathbb{Z}}
\newcommand{\F}{\mathbb{F}}
\newcommand{\K}{\mathbb{K}}
\newcommand{\N}{\mathbb{N}}
\newcommand{\E}{\mathbb{E}}
\newcommand{\fs}{\mathscr{S}} %fancy S
\newcommand{\ff}{\mathscr{F}} %fancy F
\newcommand{\OO}{\mathcal{O}}
\newcommand{\FF}{\mathcal{F}}
\newcommand{\bs}{\mathbb{S}}


\newcommand{\fg}{\mathfrak{g}}
\newcommand{\LL}{\mathcal{L}}
\newcommand{\fke}{\mathfrak{e}}
\newcommand{\al}{\alpha}
\newcommand{\ga}{\gamma}
\newcommand{\de}{\delta}
\newcommand{\Ga}{\Gamma}
\newcommand{\be}{\beta}
\newcommand{\Lm}{\Lambda}
\newcommand{\lm}{\lambda}
\newcommand{\Sig}{\Sigma}
\newcommand{\sig}{\sigma}
\newcommand{\Tht}{\Theta}
\newcommand{\tht}{\theta}
\newcommand{\vphi}{\varphi}
\newcommand{\vep}{\varepsilon}



%========= Linear Algebra
\newcommand{\bpm}{\begin{pmatrix}}
\newcommand{\epm}{\end{pmatrix}}
\newcommand{\bsm}{\left( \begin{smallmatrix}}
\newcommand{\esm}{\end{smallmatrix}\right)}
\newcommand{\hh}{\hspace{2em}}
\newcommand{\vect}{\overset{\rightharpoonup}}


%========= Algebra notation
\newcommand{\Aut}{\text{Aut}}
\newcommand{\Inn}{\text{Inn}}
\newcommand{\Tor}{\text{Tor}}
\newcommand{\Hom}{\text{Hom}}
\newcommand{\End}{\text{End}}
\newcommand{\Gal}{\text{Gal}}
\newcommand{\Fix}{\text{Fix}}


%========= Analysis notation
\newcommand{\M}{\mathcal{M}} 



%========= Topology notation
\newcommand{\YY}{\mathcal{Y}}
\newcommand{\PP}{\mathcal{P}}
\newcommand{\BB}{\mathcal{B}}
\newcommand{\CS}{\mathcal{S}}
\newcommand{\CC}{\mathcal{C}}
\newcommand{\FB}{\mathfrak{B}}
\newcommand{\EE}{\mathcal{E}}
\newcommand{\wt}{\widetilde}
\newcommand{\es}{\varnothing}



%========= Diff. Geo Shorthand
\newcommand{\bv}{\textbf{v}}
\newcommand{\bw}{\textbf{w}}
\newcommand{\A}{\mathcal{A}}
\newcommand{\BS}{\mathbb{S}}
\newcommand{\BSS}{\mathbb{S}^1}
\newcommand{\BSN}{\mathbb{S}^n}
\newcommand{\CP}{\mathbb{CP}}
\newcommand{\B}{\mathbb{B}}
\newcommand{\RP}{\mathbb{RP}}
\newcommand{\Cin}{C^\infty}
\newcommand{\px}{\widehat{x}}
\newcommand{\lh}  %left hook
{\mathbin{\mathpalette\blh\relax}}
\newcommand{\blh}[2]{\raisebox{\depth}{\scalebox{1}[-1]{$#1\lnot$}}} 





\titleformat{\section}[hang]{\centering\large\bfseries}{\thesection.}{1em}{}
\titleformat{\subsection}[runin]{\large\itshape}{- }{0em}{}[ -]

\fancyhf{} %these three lines put the page number at the bottom right
\rfoot{\thepage}
\renewcommand{\headrulewidth}{0pt}



\pagestyle{fancy}

\renewcommand{\qedsymbol}{$\clubsuit$}


\begin{document}

\setcounter{section}{8}
\section{Polynomial Rings}
\setcounter{thm}{0}

\begin{prop}
Let $I$ be an ideal of $R$ and let $(I) = I[x]$ denote the ideal of $R[x]$ generated by $I$. Then 
\[R[x]/(I) \cong (R/I)[x].\]
In particular, if $I$ is a prime ideal of $R$ then (I) is a prime ideal of $R[x]$
\end{prop}

\nl

\begin{defn}
The \textit{polynomial ring in the variables} $x_1,x_2,\ldots,x_n$ \textit{with coefficients in $R$}, denoted $R[x_1,x_2,\ldots,x_n]$, is defined inductively by
\[R[x_1,x_2,\ldots,x_n] = R[x_1,x_2,\ldots,x_{n-1}][x_n]\]
\end{defn}

\nl

\begin{thm}
Let $F$ be a field. The polynomial ring $F[x]$ is a Euclidean Domain. Specifically, if $a(x)$ and $b(x)$ are two polynomials in $F[x]$ with $b(x)$ nonzero, the there are \textit{unique} $q(x)$ and $r(x)$ in $F[x]$ such that
\[a(x) = q(x)b(x) + r(x)\qquad\text{with } r(x) = 0\text{ or } deg(r(x))<deg(b(x)).\]
\end{thm}

\nl

\begin{prop}\textit{(Gauss' Lemma)} Let $R$ be a UFD with field of fractions $F$ and let $p(x)\in R[x]$. If $p(x)$ is reducible in $F[x]$ then $p(x)$ is reducible in $R[x]$. More precisely, if $p(x) = A(x)B(x)$ for some nonconstant polynomials $A(x),B(x)\in F[x]$, then there are some nonzero elements $r,s\in F$ such that $rA(x) = a(x)$ and $sB(x) = b(x)$ both lie in $R[x]$ and $p(x) = a(x)b(x)$ is a factorization in $R[x]$.
\end{prop}

\nl

\begin{cor}
Let $R$ be a UFD, let $F$ be its field of fractions and let $p(x)\in R[x]$. Suppose the gcd of the coefficients of $p(x)$ is 1. Then $p(x)$ is irreducible in $R[x]$ if and only if it is irreducible in $F[x]$. In particular, if $p(x)$ is a monic polynomial that is irreducible in $R[x]$, then $p(x)$ is irreducible in $F[x]$.
\end{cor}

\nl

\begin{thm}
$R$ is a UFD if and only if $R[x]$ is a UFD.
\end{thm}

\nl

\begin{cor}
If $R$ is a UFD, then a polynomial ring in an arbitrary number of variables with coefficients in $R$ is also a UFD.
\end{cor}

\nl

\begin{prop}
Let $F$ be a field and let $p(x)\in F[x]$. Then $p(x)$ has a factor of degree one if and only if $p(x)$ has a root in $F$.
\end{prop}

\nl

\begin{prop}
A polynomial of degree two or three is reducible over a field $F$ if and only if it has a root in $F$.
\end{prop}

\nl

\begin{prop}
Let $p(x) = a_nx^n+ a_{n-1}x^{n-1} + \cdots +a_0$ be a polynomial with integer coefficients. If $r/s\in \Q$ is in lowest terms and $r/s$ is a root of $p(x)$, then $r$ divides the constant term and $s$ divides the leading coefficient of $p(x)$. In particular, if $p(x)$ is a monic polynomial with integer coefficients and $p(d) \neq 0$ for all integers dividing the constant term of $p(x)$, then $p(x)$ has no roots in $\Q$.
\end{prop}

\nl

\begin{prop}
Let $I$ be a proper ideal in the integral domain $R$ and let $p(x)$ be a nonconstant monic polynomial in $R[x]$. If the image of $p(x)$ in $(R/I)[x]$ cannot be factored in $(R/I)[x]$ into two polynomials of smaller degree, then $p(x)$ is irreducible in $R[x]$.
\end{prop} 

\nl

\begin{prop}\textit{(Eisenstein's Criterion)}
Let $P$ be a prime ideal of the integral domain $R$ and let $f(x) = x^n + a_{n-1}x^{n-1}+\cdots + a_0$ be a polynomial in $R[x]$ where $n\geq 1$. Suppose $a_{n-1}, \ldots a_0$ are all elements of $P$ and suppose $a_0$ is not an element of $P^2$. Then $f(x)$ is irreducible in $R[x]$.
\end{prop}

\nl

\begin{prop}
The maximal ideals in $F[x]$ are the ideals $(f(x))$ generated by irreducible polynomials $f(x)$. In particular $F[x]/ (f(x))$ is a field if and only if $f(x)$ is irreducible.
\end{prop}

\nl

\begin{prop}
Let $g(x)$ be a nonconstant monic element of $F[x]$ and let
\[g(x) = f_1(x)^{n_1}f_2(x)^{n_2}\cdots f_k(x)^{n_k}\]
be its factorization into irreducible, where the $f_i(x)$ are distinct. Then we have the following isomorphism of rings:
\[F[x]/(g(x)) \cong F[x]/ (f_1(x)^{n_1}) \times F[x]/ (f_2(x)^{n_2}) \times \cdots F[x]/ (f_k(x)^{n_k}).\]
\end{prop}

\nl

\begin{prop}
If the polynomial $f(x)$ has roots $\al_1,\al_2,\ldots,\al_k$ in $F$, then $f(x)$ has $(x-\al_1)\cdots (x-\al_k)$ as a factor. In particular, a polynomial of degree $n$ in one variable has at most $n$ roots in $F$, even counted with multiplicity.
\end{prop}

\nl

\begin{prop}
A finite subgroup of the multiplicative group of a field is cyclic. In particular, if $F$ is a finite field, the the multiplicative group $F^\times$ of nonzero elements of $F$ is a cyclic group.
\end{prop}

\nl

\begin{cor}
Let $n\geq 2$ be an integer with factorization $n = p_1^{\al_1}p_2^{\al_2}\cdots p_r^{\al_r}$ in $\Z$, where $p_1,p_2,\ldots,p_r$ are distinct primes. We have the following isomorphism of (multiplicative) groups:
\begin{enumerate}
\item $(\Z/n\Z)\cong (\Z/p_1^{\al_1}\Z)^\times \times (\Z/p_2^{\al_2}\Z)^\times \times \cdots \times(\Z/p_r^{\al_r}\Z)^\times$
\item $(\Z/2^\al \Z)^\times$ is the direct product of a cyclic group of order 2 and a cyclic group of order $2^{\al -2}$, for all $\al \geq 2$
\item $(\Z/p^\al \Z)^\times$ is a cyclic group of order $p^{\al -1}(p-1)$, for all odd primes $p$.
\end{enumerate}
\end{cor}


\end{document}
