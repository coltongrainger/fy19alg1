\documentclass{article}
\usepackage[T1]{fontenc}
\usepackage[english]{babel}
\usepackage[utf8]{inputenc}
\usepackage{fancyhdr}
\usepackage{amsmath}
\usepackage{amsfonts}
\usepackage{amssymb}
\usepackage{amsthm} 
\usepackage{thmtools}
\usepackage{lipsum}
\usepackage{geometry}
\usepackage{mathtools}
\usepackage{bold-extra}
\usepackage{mathrsfs}
\usepackage{tikz}
\usepackage{tikz-cd}
\usepackage[makeroom]{cancel}
\usepackage{hanging}
\usepackage{stmaryrd}
\usepackage{enumerate}
\usepackage{color, soul}
\usepackage{fancyhdr}
\usepackage{titlesec}
\usepackage{parskip}
\usepackage{soul}
\usepackage{graphicx}
\usepackage{mathdots}

\DeclareSymbolFont{extraup}{U}{zavm}{m}{n}
\DeclareMathSymbol{\varheart}{\mathalpha}{extraup}{86}
\DeclareMathSymbol{\vardiamond}{\mathalpha}{extraup}{87}

\geometry{a4paper,total={168mm,248mm},left=21mm,top=22mm}

%theorems, etc.
\theoremstyle{definition}
\newtheorem{thm}{Theorem}[section]
%\numberwithin{thm}{subsection}
\newtheorem{lem}[thm]{Lemma}
%\numberwithin{lem}{subsection}
\newtheorem{prop}[thm]{Proposition}
%\numberwithin{prop}{subsection}
\newtheorem{cor}[thm]{Corollary}
%\numberwithin{cor}{subsection}
\newtheorem{defn}[thm]{Definition}
%\numberwithin{defn}{subsection}
\newtheorem{conj}[thm]{Conjecture}
%\numberwithin{conj}{subsection}
\newtheorem{exam}[thm]{Example}
%\numberwithin{exam}{subsection}
\newtheorem{alg}[thm]{Algorithm}
%\numberwithin{alg}{subsection}
\newtheorem{hw}[thm]{Exercise}
\newtheorem{note}[thm]{Note}
%\numberwithin{note}{subsection}
\newtheorem{rem}[thm]{Remark}
%\numberwithin{rem}{subsection}

\newcommand{\dfn}{\textbf{Definition. }}


%\numberwithin{equation}{section}

\usepackage[colorlinks]{hyperref}
\usepackage[nameinlink,capitalize]{cleveref}


%\titleformat{<command>}[<shape>]{<format>}{<label>}{<sep>}{<before-code>}[<after-code>]

\setlength\parindent{0pt}


%========= Spacing and format
\newcommand{\cen}{\centerline}
\newcommand{\hang}{\hangindent=0.8cm}
\newcommand{\nhang}{\hangindent=0cm}
\newcommand{\nf}{\normalfont}
\newcommand{\fl}{\noindent}
\newcommand{\vs}{\vspace{0.7em}}
\newcommand{\vv}{\\\vs}
\newcommand{\nl}{\textcolor{white}{nothing}}



%========= Common Math commands
\newcommand{\ex}{\exists}
\newcommand{\nin}{\not\in}
\newcommand{\ra}{\rightarrow}
\newcommand{\Ra}{\Rightarrow}
\newcommand{\La}{\Leftarrow}
\newcommand{\oa}{\overrightarrow}
\newcommand{\lbb}{\llbracket}
\newcommand{\rbb}{\rrbracket}
\newcommand{\p}{\prime}
\newcommand{\wh}{\widehat}
\newcommand{\os}{\overset}
\newcommand{\us}{\underset}
\newcommand{\mf}{\mathfrak}
\newcommand{\ol}{\overline}
\newcommand{\td}{\widetilde}
\newcommand{\seq}{\subseteq}
\newcommand{\lp}{\left(}
\newcommand{\rp}{\right)}
\newcommand{\im}{\text{im}}
\newcommand{\inv}{^{-1}}




%========= Math letters
\newcommand{\R}{\mathbb{R}}
\newcommand{\C}{\mathbb{C}}
\newcommand{\Q}{\mathbb{Q}}
\newcommand{\Z}{\mathbb{Z}}
\newcommand{\F}{\mathbb{F}}
\newcommand{\K}{\mathbb{K}}
\newcommand{\N}{\mathbb{N}}
\newcommand{\E}{\mathbb{E}}
\newcommand{\fs}{\mathscr{S}} %fancy S
\newcommand{\ff}{\mathscr{F}} %fancy F
\newcommand{\OO}{\mathcal{O}}
\newcommand{\FF}{\mathcal{F}}
\newcommand{\bs}{\mathbb{S}}


\newcommand{\fg}{\mathfrak{g}}
\newcommand{\LL}{\mathcal{L}}
\newcommand{\fke}{\mathfrak{e}}
\newcommand{\al}{\alpha}
\newcommand{\ga}{\gamma}
\newcommand{\de}{\delta}
\newcommand{\Ga}{\Gamma}
\newcommand{\be}{\beta}
\newcommand{\Lm}{\Lambda}
\newcommand{\lm}{\lambda}
\newcommand{\Sig}{\Sigma}
\newcommand{\sig}{\sigma}
\newcommand{\Tht}{\Theta}
\newcommand{\tht}{\theta}
\newcommand{\vphi}{\varphi}
\newcommand{\vep}{\varepsilon}



%========= Linear Algebra
\newcommand{\bpm}{\begin{pmatrix}}
\newcommand{\epm}{\end{pmatrix}}
\newcommand{\bsm}{\left( \begin{smallmatrix}}
\newcommand{\esm}{\end{smallmatrix}\right)}
\newcommand{\hh}{\hspace{2em}}
\newcommand{\vect}{\overset{\rightharpoonup}}


%========= Algebra notation
\newcommand{\Aut}{\text{Aut}}
\newcommand{\Inn}{\text{Inn}}
\newcommand{\Tor}{\text{Tor}}
\newcommand{\Hom}{\text{Hom}}
\newcommand{\End}{\text{End}}
\newcommand{\Gal}{\text{Gal}}
\newcommand{\Fix}{\text{Fix}}


%========= Analysis notation
\newcommand{\M}{\mathcal{M}} 



%========= Topology notation
\newcommand{\YY}{\mathcal{Y}}
\newcommand{\PP}{\mathcal{P}}
\newcommand{\BB}{\mathcal{B}}
\newcommand{\CS}{\mathcal{S}}
\newcommand{\CC}{\mathcal{C}}
\newcommand{\FB}{\mathfrak{B}}
\newcommand{\EE}{\mathcal{E}}
\newcommand{\wt}{\widetilde}
\newcommand{\es}{\varnothing}



%========= Diff. Geo Shorthand
\newcommand{\bv}{\textbf{v}}
\newcommand{\bw}{\textbf{w}}
\newcommand{\A}{\mathcal{A}}
\newcommand{\BS}{\mathbb{S}}
\newcommand{\BSS}{\mathbb{S}^1}
\newcommand{\BSN}{\mathbb{S}^n}
\newcommand{\CP}{\mathbb{CP}}
\newcommand{\B}{\mathbb{B}}
\newcommand{\RP}{\mathbb{RP}}
\newcommand{\Cin}{C^\infty}
\newcommand{\px}{\widehat{x}}
\newcommand{\lh}  %left hook
{\mathbin{\mathpalette\blh\relax}}
\newcommand{\blh}[2]{\raisebox{\depth}{\scalebox{1}[-1]{$#1\lnot$}}} 





\titleformat{\section}[hang]{\centering\large\bfseries}{\thesection.}{1em}{}
\titleformat{\subsection}[runin]{\large\itshape}{- }{0em}{}[ -]

\fancyhf{} %these three lines put the page number at the bottom right
\rfoot{\thepage}
\renewcommand{\headrulewidth}{0pt}



\pagestyle{fancy}

\renewcommand{\qedsymbol}{$\clubsuit$}


\begin{document}

\section{Introduction to Rings}

\setcounter{thm}{0}

\begin{defn}\nl
\begin{enumerate}
\item a ring $R$ is a set together with two binary operations $+$ and $\times$ satisfying the following axioms
\begin{enumerate}
\item $(R, +)$ is an abelian group 
\item $\times$ is associative 
\item the distributive laws hold in $R$
\end{enumerate}
\item The ring $R$ is commutative if $\times$ is commutative
\item The ring $R$ is said to have identity if there is an element $1\in R$.
\end{enumerate}

\end{defn}

\nl

\begin{defn}
A ring with identity $R$ is said to be a \textit{division ring} if very nonzero element has a multiplicative inverse. A commutative division ring is called a \textit{field}.
\end{defn}

\nl

\begin{defn}\nl
\begin{enumerate}
\item A nonzero element $a$ of $R$ is called a \textit{zero divisor} if there is a nonzero element $b\in R$ such that $ab = 0$ or $ba = 0$.
\item Assume that $R$ has identity $1\neq 0$. An element $u$ of $R$ is called a \textit{\textbf{unit}} in $R$ if there is some $v$ in $R$ such that $uv=vu=1$. The set of units is denoted $R^\times$.
\end{enumerate}
\end{defn}

\nl

\begin{defn}
\hl{A commutative ring with identity is called an \textit{\textbf{integral domain}} if it has no zero divisors.}
\end{defn}

\nl

\begin{prop}
Assume that $a,b,$ and $c$ are elements of any ring with $a$ not a zero divisor. If $ab=ac$ then either $a=0$ or $b=c$.
\end{prop}

\nl

\begin{cor}
Any finite integral domain is a field.
\end{cor}

\nl

\begin{defn}
A \textit{subring} of the ring $R$ is \hl{a subgroup of $R$ that is closed under multiplication.}
\end{defn}

\nl

\begin{prop}
Let $R$ be an integral domain and let $p(x), q(x)$ be nonzero elements of $R[x]$. Then
\begin{enumerate}
\item $deg(p(x)q(x)) = deg (p(x)) + deg(q(x))$,
\item the units of $R[x]$ are just the units of $R$,
\item $R[x]$ is an integral domain.
\end{enumerate}
\end{prop}

\nl

\begin{defn}
Let $R$ and $S$ be rings.
\begin{enumerate}
\item A \textit{ring homomorphism} is a map $\vphi: R\ra S$ satisfying
\begin{enumerate}
\item $\vphi(a+b) = \vphi(a)+\vphi(b)$ for all $a,b\in R$, and 
\item $\vphi(ab) = \vphi(a)\vphi(b)$ for all $a,b\in R$
\end{enumerate}
\item The \textit{kernel} of the ring homomorphism $\vphi$ is the set of elements that map to $0_S$.
\item A bijective ring homomorphism is called an isomorphism.
\end{enumerate}
\end{defn}

\nl

\begin{prop}
Let $R$ and $S$ be rings and let $\vphi:R\ra S$ be a homomorphism.
\begin{enumerate}
\item The image of $\vphi$ is a subring of $S$.
\item The kernel of $\vphi$ is a subring of $R$. Furthermore, \hl{if $\al\in\ker(\vphi)$ then $r\al$ and $\al r$ are in $\ker(\vphi)$ for every $r\in R$.}
\end{enumerate}
\end{prop}

\nl

\begin{defn}
Let $R$ be a ring, let $I$ be a subset of $R$ and let $r\in R$.
\begin{enumerate}
\item $rI = \{ra\ |\ a\in I\}$
\item A subset $I$ of $R$ is a \textit{\textbf{left ideal}} of $R$ if 
\begin{enumerate}
\item $I$ is a subring of $R$, and
\item $I$ is closed under left multiplication by elements from $R$, i.e., $rI\seq I$ for all $r\in R$.
\end{enumerate}
There is a similar definition for a right ideal.
\item A subset $I$ that is both a left ideal and a right ideal is called an ideal of $R$.
\end{enumerate}
\end{defn}

\nl

\begin{prop}
Let $R$ be a ring and let $I$ be an ideal of $R$. Then the (additive) quotient group $R/I$ is a ring under the binary operations:
\[(r+I)+(s+I) = (r+s)+I\qquad\text{and}\qquad(r+I)\times(s+I)=(rs)+I\]
for all $r,s\in R$. Conversely if $I$ is any subgroup such that the above operations are well defined, then $I$ is an ideal of $R$.
\end{prop}

\nl

\begin{defn}
When $I$ is an ideal of $R$ the ring $R/I$ with the operations in the previous proposition is called the \textit{\textbf{quotient ring}} of $R$ by $I$.
\end{defn}

\nl

\begin{thm}\nl
\begin{enumerate}
\item \textit{(The First Isomorphism Theorem for Ring)} If $\vphi:R\ra S$ is a homomorphism of rings, then the kernel of $\vphi$ is an ideal of $R$, the image of $\vphi$ is a subring of $S$, and $R/\ker(\vphi)$ is isomorphic as a ring to $\vphi(R)$.
\item If $I$ is any ideal of $R$, then the map 
\[R\ra R/I\qquad\text{defined by}\qquad r\mapsto r+I\]
is a surjective homomorphism with kernel $I$. Thus every ideal is the kernel of a ring homomorphism and vice versa.
\end{enumerate}
\end{thm}

\nl

\begin{thm}\nl
\begin{enumerate}
\item \textit{(The Second Isomorphism Theorem for Rings)} Let $A$ be a subring and let $B$ be an ideal of $R$. Then $A+B = \{a + b\ |\ a\in A, \ b\in B\}$ is a subring of $R$, $A\cap B$ is an ideal of $A$, and $(A+B)/B\cong A/(A\cap B)$.
\item \textit{(The Third Isomorphism Theorem for Rings)} Let $I$ and $J$ be ideals of $R$ with $I\seq J$. Then $J/I$ is an ideal of $R/I$ and $(R/I)/(J/I)\cong R/J$.
\item \textit{(The Fourth or Lattice Isomorphism Theorem for Rings)} Let $I$ be an ideal of $R$. The correspondence $A\leftrightarrow A/I$ is an inclusion preserving bijection between the set of subrings $A$ of $R$ that contain $I$ and the set of subrings of $R/I$ Furthermore, $A$ is an ideal of $R$ if and only if $A/I$ is an ideal of $R/I$.
\end{enumerate}
\end{thm}

\nl

\begin{defn}
Let $R$ be a ring. Then the \hl{\textit{\textbf{characteristic}}} of the ring $R$ is the smallest number $n$ such that $n1 = 1+1+1+\cdots+1 = 0$. If this never happens, then the characteristic of $R$ is said to be $0$.
\end{defn}

\nl

\begin{prop}
\hl{Let $R$ be an integral domain. Then $char(R)$ is either prime or 0.}
\end{prop}

\nl

\begin{defn}
Let $A$ be any subset of the ring $R$.
\begin{enumerate}
\item Let $(A)$ denote the smallest ideal of $R$ containing $A$, called \textbf{\textit{the ideal generated by $A$}}.
\item Let $RA$ denote the set of all finite sums of elements of the form $ra$ with $r\in R$ and $a\in A$.
\item \hl{An ideal generated by a single element is called a \textit{\textbf{principal ideal}}.}
\item An ideal generated by a finite set is called a \textbf{\textit{finitely generated ideal}}.
\end{enumerate}
\end{defn}

\nl

\begin{prop}
Let $I$ be an ideal of $R$.
\begin{enumerate}
\item \hl{$I = R$ if and only if $I$ contains a unit.}
\item Assume $R$ is commutative. Then $R$ is a field if and only if its only ideals are $0$ and $R$.
\end{enumerate}
\end{prop}

\nl

\begin{cor}
If $R$ is a field then any nonzero ring homomorphism from $R$ into another ring is an injection (the kernel of the ring homomorphism is an ideal).
\end{cor}

\nl

\begin{defn}
An ideal $M$ in an arbitrary ring $S$ is called a \textbf{\textit{maximal ideal}} if $M\neq S$ and the only ideals containing $M$ are $M$ and $S$.
\end{defn}

\nl

\begin{prop}
In a ring with identity every proper ideal is contained in a maximal ideal. [NB: This is important because this means ideals in a ring with identity satisfy the ascending chain condition. This becomes really important in the study of infinite rings like the power series ring $\Z\lbb x\rbb$.]
\end{prop}

\nl

\begin{prop}
\hl{Assume $R$ is commutative. The ideal $M$ is maximal if and only if the quotient ring $R/M$ is a field.}
\end{prop}

\nl

\begin{defn}
Assume $R$ is commutative. An ideal $P$ is called a \hl{\textit{\textbf{prime ideal}}} if $P\neq R$ and whenever the product $ab$ of two elements $a,b\in R$ is an element of $P$, then at least one of $a$ and $b$ is an element of $P$.
\end{defn}

\nl

\begin{prop}
\hl{Assume $R$ is commutative. Then the ideal $P$ is a prime ideal in $R$ if and only if the quotient ring $R/P$ is an integral domain.}
\end{prop}

\nl

\begin{cor}
Assume $R$ is commutative. Every maximal ideal of $R$ is a prime ideal.
\end{cor}

\nl

\begin{thm}
Let $R$ be a commutative ring. Let $D$ be any nonempty subset of $R$ that does not contain 0, does not contain any zero divisors, and is closed under multiplication. Then there is a commutative ring $Q$ with 1 such that $Q$ contains $R$ as a subring and every element of $D$ is a unit in $Q$. The ring $Q$ has the following additional properties:
\begin{enumerate}
\item every element of $Q$ is of the form $rd\inv$ for some $r\in R$ and $d\in D$. In particular, if $D = R\backslash\{0\}$ then $Q$ is a field.
\item (uniqueness of $Q$) The ring $Q$ is the \textit{"smallest"} ring containing $R$ in which all the elements of $D$ become units, in the following sense. Let $S$ be any commutative ring with identity and let $\vphi:R\ra S$ be any injective ring homomorphism such that $\vphi(d)$ is a unit in $S$ for every $d\in D$. Then there is an injective homomorphism $\Phi:Q\ra S$ such that $\Phi|_R = \vphi$. In other words, any ring containing an isomorphic copy of $R$ in which all the elements of $D$ become units must also contain an isomorphic copy of $Q$.
\end{enumerate}
\end{thm}

\nl

\begin{defn}
Let $R$, $D$, and $Q$ be as in the above theorem.
\begin{enumerate}
\item The ring $Q$ is called the \textit{\textbf{ring of fractions}} of $D$ with respect to $R$ and is denoted $D\inv R$.
\item If $R$ is an integral domain and $D = R\backslash\{0\}$, $Q$ is called the \textit{\textbf{field of fractions}} or \textit{quotient field} of $R$.
\end{enumerate}
\end{defn}

\nl

\begin{cor}
Let $R$ be an integral domain and let $Q$ be the field of fractions of $R$. If a field $F$ contains a subring $R^\p$ isomorphic to $R$ then the subfield of $F$ generated by $R^\p$ is isomorphic to $Q$.
\end{cor}

\nl

\begin{defn}
The ideals $A$ and $B$ of the ring $R$ are said to be \textbf{\textit{comaximal}} if $A + B = R$.
\end{defn}

\nl

\begin{thm}\hl{\textit{(Chinese Remainder Theorem)}} Let $A_1,A_2,\ldots,A_k$ be ideals in $R$. The map
\[R\ra R/A_1\times R/A_2\times\cdots\times R/A_k\quad\text{defined by}\quad r\mapsto(r+A_1,r+A_2,\ldots,r+A_k)\]
is a ring homomorphism with kernel $\cap A_i$. If for each $i,j\in\{1,2,\ldots,k\}$ with $i\neq j$ the ideals $A_i$ and $A_j$ are comaximal, then this map is surjective and $A_1\cap A_2\cap\cdots\cap A_k = A_1A_2\cdots A_k$, so
\[R/(A_1A_2\cdots A_k) = R/(A_1\cap A_2\cap\cdots\cap A_k) \cong R/A_1\times R/A_2\times\cdots\times R/A_k.\]
\end{thm}

\nl

\begin{cor}
Let $n$ be a positive integer and let $p_1^{\al_1}p_2^{\al_2}\cdots p_k^{\al_k}$ be its factorization into powers of distinct primes. Then
\[Z/n\Z \cong (\Z/p_1^{\al_1}\Z)\times(\Z/p_2^{\al_2}\Z)\times\cdots\times (\Z/p_k^{\al_k}\Z),\]
as rings, so in particular we have the following isomorphism of multiplicative groups:
\[(Z/n\Z)^\times \cong (\Z/p_1^{\al_1}\Z)^\times\times(\Z/p_2^{\al_2}\Z)^\times\times\cdots\times (\Z/p_k^{\al_k}\Z)^\times.\]
\end{cor}

\nl

\begin{cor}
Let $a,b\in \Z$ then
\[\Z/(m)\times \Z/(n) \cong \Z/(\gcd(m,n))\times \Z/(\text{lcm}(m,n))\]
\end{cor}

\section{Euclidean Domains, Principal Ideal Domains, and Unique Factorization Domains}

\hl{All rings in this section are commutative.}

\nl

\begin{defn}
Any function $N:R\ra \Z_{\geq 0}$ with $N(0) = 0$ is called a \textit{\textbf{norm}} on the integral domain $R$. If $N(a)> 0$ for all $a\neq 0$ define $N$ to be a \textit{positive norm}.
\end{defn}

\nl

\begin{defn}
The integral domain $R$ is said to be a \hl{\textit{\textbf{Euclidean Domain}}} if there is a norm $N$ on $R$ such that for any two elements $a$ and $b$ of $R$ with $b\neq 0$ there exist elements $q$ and $r$ in $R$ with 
\[a = qb + r\qquad \text{with } r = 0 \text{ or } N(r)<N(b).\]
\end{defn}

\nl

\begin{defn}
Let $R$ be a commutative ring and let $a,b\in R$ with $b\neq 0$.
\begin{enumerate}
\item $a$ is said to be a \textbf{\textit{multiple}} of $b$ if $a = bx$ for some $x\in R$. In this case $b$ is said to divide or be a divisor of $a$, written $b\ |\ a$.
\item A \textbf{\textit{greatest common divisor}} of $a$ and $b$ is a nonzero element $d$ such that 
\begin{enumerate}
\item $d\ |\ a$ and $d\ |\ b$, and 
\item if $d^\p\ |\ a$ and $d^\p\ |\ b$ then $d\ |\ d^\p$.
\end{enumerate}
A greatest common divisor of $a$ and $b$ will be denoted by $\gcd(a,b)$.
\end{enumerate}
\end{defn}

\nl

\begin{prop}
If $a$ and $b$ are nonzero elements in the commutative ring $R$ such that the ideal generated by $a$ and $b$ is a principal ideal $(d)$, then $d$ is a greatest common divisor of $a$ and $b$.
\end{prop}

\nl

\begin{prop}
Let $R$ be an integral domain. If two elements $d$ and $d^\p$ of $R$ generate the same principal ideal, then $d^\p = ud$ for some unit $u\in R$. In particular, if $d$ and $d^\p$ are both greatest common divisors of $a$ and $b$, then $d^\p = ud$ for some unit $u$.
\end{prop}

\nl

\begin{thm}
Let $R$ be a Euclidean Domain and let $a$ and $b$ be nonzero elements of $R$. Let $d = r_n$ be the last nonzero remainder in the Euclidean Algorithm for $a$ and $b$. Then
\begin{enumerate}
\item $d$ is a greatest common divisor of $a$ and $b$, and 
\item the principal ideal $(d)$ is the ideal generated by $a$ and $b$. In particular, $d$ can be written as an $R$\textbf{\textit{-linear combination}} of $a$ and $b$, i.e., there are elements $x$ and $y$ in $R$ such that 
\[d = ax + by.\]
\end{enumerate}
\end{thm}

\nl

\begin{defn}
A domain $R$ in which every ideal is principal is called a \textbf{\textit{Principal Ideal Domain}} (PID).
\end{defn}

\nl

\begin{prop}
Let $R$ be a PID and let $a$ and $b$ be nonzero elements of $R$. Let $d$ be a generator for the principal ideal generated by $a$ and $b$. Then
\begin{enumerate}
\item $d$ is a greatest common divisor of $a$ and $b$
\item $d$ can be written as an $R$-\textit{linear combination} of $a$ and $b$, i.e., there are elements $x$ and $y$ in $R$ with 
\[d = ax + by\]
\item $d$ is unique up to multiplication by a unit in $R$.
\end{enumerate}
\end{prop}

\nl

\begin{prop}
\hl{Every nonzero prime ideal in a PID is a maximal ideal.} 
\end{prop}

\nl

\begin{cor}
If $R$ is any commutative ring such that the polynomial ring $R[x]$ is a PID (or Euclidean Domain), then $R$ is necessarily a field.
\end{cor}

\nl

\begin{defn}
Let $R$ be an integral domain
\begin{enumerate}
\item Suppose $r\in R$ is nonzero and is not a unit. Then $r$ is called \textit{\textbf{irreducible}} if $R$ if whenever $r = ab$ with $a,b\in R$ at least one of $a$ or $b$ is a unit in $R$.
\item The nonzero element $p\in R$ is called \textbf{\textit{prime}} in $R$ it the ideal $(p)$ generated by $p$ is a prime ideal. In other words, for any $a,b\in R$ if $p\ |\ ab$ then either $p\ |\ a$ or $p\ |\ b$.
\item Two elements $a,b\in R$ differing by a unit are said to be \textit{\textbf{associate}} in $R$.
\end{enumerate}
\end{defn}

\nl

\begin{prop}
\hl{In an integral domain a prime element is always irreducible.}
\end{prop}

\nl

\begin{prop}
In a PID a nonzero element is prime if and only if it is irreducible.
\end{prop}

\nl

\begin{defn}
A \hl{\textit{\textbf{Unique Factorization Domain (UFD)}}} is an integral domain $R$ in which every nonzero element $r\in R$ which is not a unit has the following two properties:
\begin{enumerate}
\item $r$ can be written as the finite product of irreducibles $p_i$ of $R$: $r = p_1p_2\cdots p_n$ and
\item the decomposition given in (1) is unique up to associates. 
\end{enumerate}
\end{defn}

\nl

\begin{prop}
\hl{In a UFD a nonzero element is a prime if and only if it is irreducible.}
\end{prop}

\nl

\begin{prop}
Let $a$ and $b$ be two nonzero elements of the UFD $R$ and suppose
\[a = u\ p_1^{e_1}p_2^{e_2}p_3^{e_3}\cdots p_n^{e_n}\qquad\text{and}\qquad b = v\ p_1^{f_1}p_2^{f_2}p_3^{f_3}\cdots p_n^{f_n}\]
are prime factorizations for $a$ and $b$, where $u$ and $v$ are units, the primes $p_1,p_2,\ldots,p_n$ are \textit{distinct} and the exponents $e_i$ and $f_i$ are $\geq 0$. Then the element 
\[d = p_1^{\min(e_1,f_1)}p_2^{\min(e_2,f_2)}p_3^{\min(e_3,f_3)}\cdots p_n^{\min(e_n,f_n)}\]
is a greatest common divisor of $a$ and $b$. 
\end{prop}

\nl

\begin{thm}
Every PID is a UFD. In particular, every Euclidean Domain is a UFD. 
\end{thm}

\nl

\begin{lem}
The prime number $p\in \Z$ divides an integer of the form $n^2 + 1$ if and only if $p$ is either 2 or is an odd prime congruent to $1\mod 4$.
\end{lem}

\nl

\begin{prop}\nl
\begin{enumerate}
\item \textit{(Fermat's Theorem on sums of squares)} The prime $p$ is the sum of two integer squares, $p = a^2 + b^2$ if and only if $p = 2$ or $p\equiv 1\mod 4$. Except for the interchanging $a$ and $b$, the representation of $p$ as the sum of two squares is unique. 
\item The irreducible elements in the Gaussian integers $\Z[i]$ are as follows
\begin{enumerate}
\item $1 + i$
\item the primes $p\in \Z$ with $p\equiv 3\mod 4$
\item $a + bi,\ a-bi$, the distinct irreducible factors of $p = a^2 + b^2$ for the primes $p\in \Z$ with $p\equiv 1\mod 4$.
\end{enumerate}
\end{enumerate}
\end{prop}


\end{document}
