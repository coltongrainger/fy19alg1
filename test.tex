\documentclass{humanist}
% pandoc is utf8 native
\usepackage[utf8]{inputenc}
% pandoc transfers h2 to subsections, I prefer h2 for sections
\let\subsubsection\subsection
\let\subsection\section
\let\section\chapter
\let\chapter\part
% section numbering
\setcounter{secnumdepth}{5}
% urls as footnotes
\usepackage[unicode=true]{hyperref}
\renewcommand{\href}[2]{#2\footnote{\url{#1}}}
% fix: parskip mangles ams table of contents
\usepackage{parskip}
\makeatletter
\renewcommand\tableofcontents{%
    \@starttoc{toc}%
}
\makeatother
% linespacing
% use upquote if available, for straight quotes in verbatim environments
\IfFileExists{upquote.sty}{\usepackage{upquote}}{}
\usepackage{booktabs,longtable}
% code blocks

%% bold math capitals
\newcommand{\BB}{\mathbf{B}}
\newcommand{\CC}{\mathbf{C}}
\newcommand{\DD}{\mathbf{D}}
\newcommand{\EE}{\mathbf{E}}
\newcommand{\FF}{\mathbf{F}}
\newcommand{\GG}{\mathbf{G}}
\newcommand{\HH}{\mathbf{H}}
\newcommand{\II}{\mathbf{I}}
\newcommand{\JJ}{\mathbf{J}}
\newcommand{\KK}{\mathbf{K}}
\newcommand{\LL}{\mathbf{L}}
\newcommand{\MM}{\mathbf{M}}
\newcommand{\NN}{\mathbf{N}}
\newcommand{\OO}{\mathbf{O}}
\newcommand{\PP}{\mathbf{P}}
\newcommand{\QQ}{\mathbf{Q}}
\newcommand{\RR}{\mathbf{R}}
\newcommand{\TT}{\mathbf{T}}
\newcommand{\UU}{\mathbf{U}}
\newcommand{\VV}{\mathbf{V}}
\newcommand{\WW}{\mathbf{W}}
\newcommand{\XX}{\mathbf{X}}
\newcommand{\YY}{\mathbf{Y}}
\newcommand{\ZZ}{\mathbf{Z}}

%% script math capitals
\newcommand{\sA}{\mathscr{A}}
\newcommand{\sB}{\mathscr{B}}
\newcommand{\sC}{\mathscr{C}}
\newcommand{\sD}{\mathscr{D}}
\newcommand{\sE}{\mathscr{E}}
\newcommand{\sF}{\mathscr{F}}
\newcommand{\sG}{\mathscr{G}}
\newcommand{\sH}{\mathscr{H}}
\newcommand{\sI}{\mathscr{I}}
\newcommand{\sJ}{\mathscr{J}}
\newcommand{\sK}{\mathscr{K}}
\newcommand{\sL}{\mathscr{L}}
\newcommand{\sM}{\mathscr{M}}
\newcommand{\sN}{\mathscr{N}}
\newcommand{\sO}{\mathscr{O}}
\newcommand{\sP}{\mathscr{P}}
\newcommand{\sQ}{\mathscr{Q}}
\newcommand{\sR}{\mathscr{R}}
\newcommand{\sS}{\mathscr{S}}
\newcommand{\sT}{\mathscr{T}}
\newcommand{\sU}{\mathscr{U}}
\newcommand{\sV}{\mathscr{V}}
\newcommand{\sW}{\mathscr{W}}
\newcommand{\sX}{\mathscr{X}}
\newcommand{\sY}{\mathscr{Y}}
\newcommand{\sZ}{\mathscr{Z}}

% more
\providecommand{\abs}[1]{\left\lvert #1 \right\rvert}
\providecommand{\norm}[1]{\left\lVert #1 \rVert\right}
\renewcommand{\phi}{\varphi}
\newcommand{\eps}{\varepsilon}
\renewcommand{\emptyset}{\varnothing}

% 2018-10-05 dropping analysis
\newcommand{\Stab}[2]{\mathrm{Stab}_{#1}\left( { #2 } \right)}
\newcommand{\Norm}[2]{\mathrm{Norm}_{#1}\left( { #2 } \right)}

% graphics
\setlength{\emergencystretch}{3em}  % prevent overfull lines
\providecommand{\tightlist}{%
\newcommand{\Stab}[2]{\mathrm{Stab}_{#1}\left( #2 \right)}
  \setlength{\itemsep}{0pt}\setlength{\parskip}{0pt}}
% set default figure placement to htbp
\makeatletter
\def\fps@figure{htbp}
\makeatother
% bibliographies

\title{Automorphisms}
\author{Colton Grainger (MATH 6130 Algebra)}
\date{2018-10-05}

\begin{document}
\maketitle

\providecommand{\Aut}[1]{\mathrm{Aut}\left({ #1 }\right)}

\setcounter{section}{5}

\hypertarget{assignment-due-2018-10-17}{%
\subsection{Assignment due 2018-10-17}\label{assignment-due-2018-10-17}}

\hypertarget{number-3.5.13df04-number-3.5.13}{%
\subsubsection{\texorpdfstring{{[}@DF04, number
3.5.13{]}}{, number 3.5.13{[}@DF04, number 3.5.13{]}}}\label{number-3.5.13df04-number-3.5.13}}

Every element of order \(2\) in \(A_n\) is the square of an element of
order \(4\) is \(S_n\). An element of order \(2\) in \(A_n\) is a
product of \(2k\) commuting transpositions.

\hypertarget{number-3.5.15df04-number-3.5.15}{%
\subsubsection{\texorpdfstring{{[}@DF04, number
3.5.15{]}}{, number 3.5.15{[}@DF04, number 3.5.15{]}}}\label{number-3.5.15df04-number-3.5.15}}

If \(x\) and \(y\) are distinct \(3\)-cycles in \(S_4\) with
\(x \neq y^{-1}\), then the subgroup of \(S_4\) generated by \(x\) and
\(y\) is \(A_4\).

\hypertarget{number-4.2.8df04-number-4.2.8}{%
\subsubsection{\texorpdfstring{{[}@DF04, number
4.2.8{]}}{, number 4.2.8{[}@DF04, number 4.2.8{]}}}\label{number-4.2.8df04-number-4.2.8}}

If \(H\) has finite index \(n\) then there is a normal subgroup \(K\) of
\(G\) with \(K \le H\) and \(\abs{G : K} \le n!\).

\hypertarget{number-4.3.13df04-number-4.3.13}{%
\subsubsection{\texorpdfstring{{[}@DF04, number
4.3.13{]}}{, number 4.3.13{[}@DF04, number 4.3.13{]}}}\label{number-4.3.13df04-number-4.3.13}}

We exhaustively list all finite groups that have exactly two conjugacy
classes.

\hypertarget{number-4.3.19df04-number-4.3.19}{%
\subsubsection{\texorpdfstring{{[}@DF04, number
4.3.19{]}}{, number 4.3.19{[}@DF04, number 4.3.19{]}}}\label{number-4.3.19df04-number-4.3.19}}

Assume \(H\) is a normal subgroup of \(G\), \(\sK\) is a conjugacy class
of \(G\) contained in \(H\) and \(x \in \sK\). We show \(\sK\) is a
union of \(k\) conjugacy classes of equal size in \(H\), where
\(k = \abs{G : HC_G(x)}\). We then show\footnote{Letting \(A = C_G(x)\)
  and \(B = H\) so \(A \cap B = C_H(x)\), drawing the lattice diagram
  associated to the second isomorphism theorem, and interpreting the
  appropriate indices.} a conjugacy class in \(S_n\) that consists of
even permutations is either a single conjugacy class under the action of
\(A_n\) or is a union of two classes of the same size in \(A_n\).

\hypertarget{number-4.3.23df04-number-4.3.23}{%
\subsubsection{\texorpdfstring{{[}@DF04, number
4.3.23{]}}{, number 4.3.23{[}@DF04, number 4.3.23{]}}}\label{number-4.3.23df04-number-4.3.23}}

If \(M\) is a maximal subgroup of \(G\) then either \(N_G(M) = M\) or
\(N_G(M) = G\). Therefore, if \(M\) is a maximal subgroup of \(G\) that
is not normal in \(G\) then the number of nonidentity elements of \(G\)
that are contained in conjugates of \(M\) is at most
\((\abs{M} -1)\cdot \abs{G:M}\).

\hypertarget{number-4.3.24df04-number-4.3.24}{%
\subsubsection{\texorpdfstring{{[}@DF04, number
4.3.24{]}}{, number 4.3.24{[}@DF04, number 4.3.24{]}}}\label{number-4.3.24df04-number-4.3.24}}

Assume \(H\) is a proper subgroup of the finite group \(G\). Then \(G\)
is not the union of the conjugates of any proper subgroup,\footnote{Hint:
  put \(H\) in some maximal subgroup and use the previous exercise.}
i.e., \[G \neq \bigcup_{g \in G} gHg^{-1}.\]

\hypertarget{the-size-of-each-conjugacy-class-in-s_n-number-4.3.33df04-number-4.3.33}{%
\subsubsection{\texorpdfstring{The size of each conjugacy class in
\(S_n\) {[}@DF04, number
4.3.33{]}}{The size of each conjugacy class in S\_n , number 4.3.33{[}@DF04, number 4.3.33{]}}}\label{the-size-of-each-conjugacy-class-in-s_n-number-4.3.33df04-number-4.3.33}}

Let \(\sigma\) be a permutation in \(S_n\) and let \(m_1, \ldots, m_s\)
be \emph{distinct} integers that appear in the cycle type of \(\sigma\)
(including \(1\)-cycles). For each \(i \in \{1, 2, \ldots, s\}\) assume
\(\sigma\) has \(k_i\) cycles of length \(m_i\) (so that
\(\sum_{i=1}^s k_i m_i = n\)). Then the number of conjugates\footnote{If
  \(n \ge m\) then the number of \(m\)-cycles in \(S_n\) is given by
  \[\frac{n(n-1)(n-2)\ldots(n-m+1)}{m}.\] For another example, if
  \(n \ge 4\) then the number of permutations in \(S_n\) that are the
  product of two disjoint \(2\)-cycles is \(n(n-1)(n-2)(n-3)/8\).} of
\(\sigma\) is
\[\frac{n!}{(k_1!m_1^{k_1})(k_2!m_2^{k_2})\cdots(k_s!m_s^{k_s})}.\]

\hypertarget{number-4.4.3df04-number-4.4.3}{%
\subsubsection{\texorpdfstring{{[}@DF04, number
4.4.3{]}}{, number 4.4.3{[}@DF04, number 4.4.3{]}}}\label{number-4.4.3df04-number-4.4.3}}

Under any automorphism of \(D_8\), \(r\) has at most \(2\) possible
images and \(s\) has at most \(4\) possible images. Thence
\(\abs{\mathrm{Aut}\left({ (D_8) }\right)} \le 8\).

\hypertarget{number-4.4.8df04-number-4.4.8}{%
\subsubsection{\texorpdfstring{{[}@DF04, number
4.4.8{]}}{, number 4.4.8{[}@DF04, number 4.4.8{]}}}\label{number-4.4.8df04-number-4.4.8}}

Suppose \(G\) is a group with subgroups \(H\) and \(K\) where
\(H \le K\).

\begin{enumerate}
\def\labelenumi{(\alph{enumi})}
\tightlist
\item
  If \(H\) is characteristic in \(K\) and \(K\) is normal in \(G\), then
  \(H\) is normal in \(G\). (\emph{optional})
\item
  If \(H\) is characteristic in \(K\) and \(K\) is characteristic in
  \(G\), then \(H\) is characteristic in \(G\). Thence the
  \emph{Viergruppe} \(V_4\) is characteristic in \(S_4\).
\item
  If \(H\) is normal in \(K\) and \(K\) is characteristic in \(G\), then
  \(H\) need not be normal in \(G\).
\end{enumerate}

\hypertarget{number-4.4.18df04-number-4.4.18}{%
\subsubsection{\texorpdfstring{{[}@DF04, number
4.4.18{]}}{, number 4.4.18{[}@DF04, number 4.4.18{]}}}\label{number-4.4.18df04-number-4.4.18}}

For \(n \neq 6\) every automorphism of \(S_n\) is inner. Fix an integer
\(n \ge 2\) with \(n \neq 6\).

\begin{enumerate}
\def\labelenumi{(\alph{enumi})}
\tightlist
\item
  The automorphism group of a group \(G\) permutes the conjugacy classes
  of \(G\), i.e., for each \(\sigma \in \mathrm{Aut}\left({ G }\right)\)
  and each conjugacy class \(\sK\) of \(G\) the set \(\sigma(\sK)\) is
  also a conjugacy class of \(G\).
\item
  Let \(\sK\) be the conjugacy class of transpositions in \(S_n\) and
  let \(\sK'\) be the conjugacy class of any element of order \(2\) in
  \(S_n\) that is not a transposition. Then
  \(\abs{\sK} \neq \abs{\sK'}\). Furthermore, any automorphism of
  \(S_n\) sends transpositions to transpositions.
\item
  For each \(\sigma \in \mathrm{Aut}\left({ S_n }\right)\) we have
  \[\sigma \colon (1\, 2) \mapsto (a\, b_2), \quad\quad \sigma \colon (1\, 3) \mapsto (a\, b_3), \quad\ldots, \quad \sigma \colon (1\, n) \mapsto (a\, b_n)\]
  for some distinct integers
  \(a, b_2, b_3, \ldots, b_n \in \{1, 2, \ldots, n\}\).
\item
  Therefore \((1\, 2), (1\, 3), \ldots, (1\, n)\) generate \(S_n\).
  Furthermore \(S_n\) is uniquely determined by its action on these
  elements. Then by (c), \(S_n\) has at \emph{most} \(n!\)
  automorphisms. We conclude that
  \(\mathrm{Aut}\left({ S_n }\right) = \mathrm{Inn}(S_n)\) for
  \(n \neq 6\).
\end{enumerate}

\hypertarget{number-4.4.20df04-number-4.4.20}{%
\subsubsection{\texorpdfstring{{[}@DF04, number
4.4.20{]}}{, number 4.4.20{[}@DF04, number 4.4.20{]}}}\label{number-4.4.20df04-number-4.4.20}}

For any finite group \(P\), let \(d(P)\) be the minimum\footnote{For
  example, \(d(P) = 1\) if and only if \(P\) is a nontrivial cyclic
  group and \(d(Q_8) = 2\).} number of generators of \(P\). Let \(m(P)\)
be the maximum of the integers \(d(A)\) as \(A\) runs\footnote{For
  example, \(m(Q_8) = 1\) and \(m(D_8) = 2\).} over all \emph{abelian}
subgroups of \(P\). Define the \emph{Thompson subgroup} of \(P\) as
\[J(P) = \langle A : A \text{ is an abelian subgroup of $P$ with } d(A) = m(P)\rangle.\]

\begin{enumerate}
\def\labelenumi{(\alph{enumi})}
\tightlist
\item
  \(J(P)\) is a characteristic subgroup of \(P\).
\item
  For each of the following groups \(P\), we exhaustively list all
  abelian subgroups \(A\) of \(P\) that satisfy \(d(A) = m(P)\).
\end{enumerate}

\begin{itemize}
\tightlist
\item
  \(Q_8\)
\item
  \(D_8\)
\item
  \(D_{16}\)
\item
  \(QD_{16}\) (the quasidihedral group of order \(16\))
\end{itemize}

\hypertarget{references}{%
\subsection{References}\label{references}}


\end{document}
