\section{Introduction to Groups}
\begin{thm}
Let $(G,\cdot)$ be a group, $H\seq G$. Then $(H,\cdot)$ is a subgroup of $G$ if
\begin{enumerate}
\item $H\neq \es$.
\item For all $a,b\in H$, $ab^{-1}\in H$.
\end{enumerate}
\end{thm}

\nl
\begin{defn}
Let $g\in G$ a group, then $g^{-1}$ is the \underline{\textit{unique}} element of $G$ such that $gg^{-1}=g^{-1}g=id$
\end{defn}

\nl
\begin{defn}
A \textbf{\textit{group action}} of a group $G$ on a set $A$ is a map, $G\times A$ to $A$ satisfying the following properties:
\begin{enumerate}
\item $g_1\cdot(g_2\cdot a) = (g_1g_2)\cdot a$, for all $g_1,g_2\in G$ and $a\in A$.
\item $id_G\cdot a = a$ for all $a\in A$.
\end{enumerate}
\end{defn}

%################################################################################

\section{Subgroups}

\setcounter{thm}{0}
\nl
\begin{defn}
Let $A\seq G$, $A\neq\es$. \hl{Define $C_G(A) = \{g\in G\ |\ gag^{-1} = a \text{ for all } a\in A\}$}. This subset of $G$ is called the \textbf{\textit{centralizer}} of $A$ in $G$. Since $gag^{-1} = a$ if and only if $ga = ag$, $C_G(A)$ is the set of elements of $G$ which commute with every element of $A$. When $A = G$ this set is denoted by $Z(G)$ and is called the \textit{\textbf{center}} of $G$.
\end{defn}
\nl

\textbf{Note:} $Z(G) \leq C_G(A)$ for all $A\seq G$.

\nl

\begin{defn}
Let $A\seq G$, $A\neq\es$. Define $gAg^{-1} = \{gag^{-1}\ |\ a\in A\}$. We define the \textit{\textbf{normalizer}} of $A$ in $G$ to be the set $N_G(A) = \{g\in G\ |\ gAg^{-1} = A\}$.
\end{defn}
\nl

\begin{defn}
Let $G$ be a group acting on a set $S$. The \textit{\textbf{stabilizer}} $G_s$ for some fixed $s\in S$ is the set 
\[G_s = \{g\in G\ |\ g\cdot s = s\}.\]
\end{defn}

\begin{prop}
Let $A$ be some set and $G$ be a group. Then $C_G(A)\leq N_G(A)$.
\end{prop}
\begin{proof}
$C_G(A)$ is the kernel of $N_G(A)$ acting on $A$ under the conjugation map $a\mapsto gag^{-1}$.
\end{proof}

\nl

\begin{prop}
Let $G$ be a group and $S\seq G$, $s\neq \es$. \hl{Then $N_G(S) \leq G$.}
\end{prop}

\begin{proof}
Let $G$ be a group and $S\seq G$. We know that $N_G(S) = \{g \in G\ |\ gSg\inv = S\}$. If we take $a, b\in N_G(S)$ we have that 
\begin{align*}
abSb\inv a\inv = aSa\inv = S
\end{align*}
so $ab \in N_G(S)$. Similarly we have that for $a\in N_G(S)$
\[a\inv Sa = a\inv(aSa\inv)a = S\]
so $a\inv \in N_G(S)$ and we have that $N_G(S) \leq G$.
\end{proof}

\nl
\begin{thm}
There is only one cyclic group of each order.
\end{thm}

\nl
\begin{prop}
Let $G$ be a group, let $x\in G$ and let $a\in \Z^\times$.
\begin{enumerate}
\item If $|x|=\infty$, then $|x^a| = \infty$.
\item If $|x| = n< \infty$, then $|x^a| = \frac{n}{\gcd(n,a)}$.
\end{enumerate}
\end{prop}

\nl
\begin{defn}
Let $A\seq G$ and define 
\[\langle A\rangle = \bigcap_{\substack{ A\seq H\\H\leq G}} H. \]
This is called the \textit{\textbf{subgroup of G generated by A}} and is simply the intersection of all the subgroups containing the set $A$.
\end{defn}


\nl

\hl{\textbf{Zorn's Lemma.}} If $A$ is a nonempty partially ordered set in which every chain has an upper bound then $A$ has a maximal element.

%################################################################################

\section{Quotient Groups and Homomorphisms}

\setcounter{thm}{0}
\begin{prop}
Let $G$ and $H$ be groups and let $\vphi: G\ra H$ be a homomorphism.
\begin{enumerate}
\item $\vphi(id_G) = id_H$
\item $\vphi(g^{-1}) = \vphi(g)^{-1}$
\item $\vphi(g^n)= \vphi(g)^n$
\item $\ker(\vphi)$ is a subgroup of $G$
\item $\im(\vphi)$ is a subgroup of $H$
\end{enumerate}
\end{prop}

\nl
\begin{prop}\label{coset_op}
Let $G$ be a group and let $N$ be a subgroup of $G$
\begin{enumerate}
\item The operation on the set of left cosets of $N$ in $G$ described by
\[uN\cdot vN = (uv)N\]
is sell defined if and only if $gng^{-1}\in N$ for all $g\in G$ and all $n\in N$.
\item If the above operation is well defined then it makes the set of left cosets of $N$ in $G$ into a group. In particular the identity of this group is the coset $id_G N$and the inverse of $gN$ is $g^{-1}N$.
\end{enumerate}
\end{prop}

\nl


\begin{defn}
The element $gng^{-1}$ is called the conjugate of $n\in N$ by $g$. The set $gNg^{-1}$ is also called the conjugate of $N$ by $g$. The element $g$ is said to \textbf{\textit{normalize}} $N$ if $gNg^{-1} = N$. A subgroup $N$ of $G$ is a \textbf{\textit{normal subgroup}} if every $g\in G$ normalizes $N$. We will write this as $N \unlhd G$.
\end{defn}

\nl

\begin{thm}
Let $N$ be a subgroup of $G$. The following are equivalent.
\begin{enumerate}
\item $N\unlhd G$
\item $N_G(N) = G$
\item $gN = Ng \quad \forall g\in G$
\item The operation on left cosets of $N$ in $G$ described by \autoref{coset_op} makes the set of left cosets into a group
\item $gNg^{-1} \seq N$ for all $g\in G$.
\end{enumerate}
\end{thm}

\hl{\textbf{Lagrange's Theorem.}} If $G$ is a finite group and $H$ is a subgroup of $G$, then the order of $H$ divides the order of $G$ and the number of left cosets of $H$ in $G$ equals $\frac{|G|}{|H|}$.

\hl{\textbf{Cauchy's Theorem.}} If $G$ s a finite group and $p$ is a prime dividing $|G|$ then $G$ has an element of order $p$.

\nl

\begin{defn}\hl{(Dedekind and Hamiltonian Groups)}
For any group $G$, if all the subgroups of $G$ are normal then $G$ is called a \textit{Dedekind} group. If $G$ is non-abelian then $G$ is called a \textit{Hamiltonian} group.
\end{defn}

\nl
\begin{thm}
If $G$ is a finite group of order $p^\al m$, where $p$ is a prime and $p$ does not divide $m$, then $G$ has a subgroup of order $p^\al$ (Proof will be done with the big Sylow theorem).
\end{thm}

\nl
\begin{defn}
Let $H$ and $K$ be subgroups of a group and define
\[HK = \{hk |\ h\in H,\ k\in K\}.\]
\end{defn}

\begin{prop}
If $H$ and $K$ are subgroups of a group then
\[|HK| = \frac{|H||K|}{|H\cap K|}.\]
\end{prop}

\begin{cor}
If $H$ and $K$ are subgroups of $G$ then $HK$ is a subgroup if $H$ normalizes $K$ (i.e. if $H\seq N_G(K)$).
\end{cor}


===============

\textbf{Isomorphism Theorems}

===============

\nl

\begin{thm}\textit{(First Isomorphism Theorem)}
If $\vphi:G \ra H$ is a homomorphism of groups, then $\ker(\vphi)\unlhd G$ and $G/\ker(\vphi) \cong \im(\vphi)$.
\end{thm}

\nl

\begin{cor}
Let $\vphi: G\ra H$ be a homomorphism of groups.
\begin{enumerate}
\item $\vphi$ is injective if and only if $\ker(\vphi) = id_G$.
\item $|G:\ker(\vphi)| = |\im(\vphi)|$.
\end{enumerate}
\end{cor}

\nl

\begin{thm}\textit{(The Second or Diamond Isomorphism Theorem)}
Let $G$ be a group, let $A$ and $B$ be subgroups of $G$ and assume $A\leq N_G(B)$. Then $AB$ is a subgroup of $G$, $B\ \unlhd\  AB,\ \ A\cap B\ \unlhd\  A$ and $AB/B\cong A/A\cap B$.
\end{thm}

\begin{center}
\begin{tikzcd}
& G \arrow[dash, d] & \\
& AB\arrow[dash, dr, "\unrhd"] \arrow[dash, dl] & \\
A\arrow[dash, dr, "\unrhd"] & & B\arrow[dash, dl] \\
& A\cap B\arrow[dash, d] & \\
& \{id_G\} &  \\
\end{tikzcd}
\end{center}


\begin{thm}\textit{(The Third Isomorphism Theorem)}
Let $G$ be a group and let $H$ and $K$ be normal subgroups of $G$ with $H\leq K$. Then $K/H\unlhd G/H$ and 
\[(G/H)/(K/H)\cong G/K\]
\end{thm}

\nl

\begin{thm}\textit{(The Fourth Isomorphism Theorem)}
Let $G$ be a group and let $N$ be a normal subgroup of $G$. The there is a bijection from the set $\CS$ of subgroups $A$ of $G$ which contain $N$ onto the set $\mathcal{T}$ of subgroups of the quotient group $G/N$. Specifically, there is a bijective map $\vphi:\CS \ra \mathcal{T}:A\mapsto A/N$ and we have the following:
\begin{enumerate}
\item $A\leq B$ if and only if $A/N\leq B/N$,
\item if $A\leq B$, then $|B:A| = |B/N:A/N|$,
\item $\langle A, B\rangle/N = \langle A/N, B/N\rangle$,
\item $(A\cap B)/N = A/N\cap B/N$, and 
\item $A\unlhd G$ if and only if $A/N \unlhd G/N$.
\end{enumerate}
\end{thm}


===============

===============

\begin{thm}\textit{(Feit-Thompson)}
If $G$ is a simple group of odd order, then $G\cong\Z/p\Z$ for some prime $p$.
\end{thm}

\nl

\begin{defn}
\hl{A group $G$ is \textbf{\textit{solvable}}} if there is a chain of subgroups 
\[1 = G_0\unlhd G_1\unlhd\cdots\unlhd G_n = G\]
such that $G_{i+1}/G_i$ is abelian for $i= 0,1,\ldots,n-1$.
\end{defn}

\nl

\begin{thm}
The finite group $G$ is solvable if and only if for every divisor $n$ of $|G|$ such that $\gcd\lp n,\frac{|G|}{n}\rp = 1$, $G$ has a subgroup of order $n$.
\end{thm}

\nl

\begin{defn}
The \textit{alternating group of degree n}, denoted by $A_n$, is the kernel of the sign homomorphism acting on $S_n$.
\end{defn}

\nl

\begin{prop}
The permutation $\sigma$ is odd if and only if the number of cycles of even length in its cycle decomposition is odd.
\end{prop}

%################################################################################

\section{Group Actions}

\setcounter{thm}{0}

\begin{defn}
Let $G$ be a group acting on a nonempty set $A$. For each $g\in G$ the map 
\[\sigma_g:A\ra A:a\mapsto g\cdot a\]
is a permutation of $A$. The homomorphism associated to an action of $G$ on $A$
\[\vphi: G\ra S_A:\vphi(g)\mapsto\sigma_g\]
is called the \textit{permutation representation} associated to the given action.
\end{defn}

\nl

\begin{defn}
Let $G$ be a group acting on a set $A$
\begin{enumerate}
\item The \textbf{\textit{kernel}} of the action is the set of elements of $G$ that act trivially on every element of $A$: $\{g\in G\ |\ g\cdot a = a\text{ for all }a\in A\}$.
\item Fro each $a\in A$ the \textbf{\textit{stabilizer}} of $a$ in $G$ is the set of elements of $G$ that fix the element $a$: $\{g\in G\ |\ g\cdot a =a\}$ and is denoted by $G_a$.
\item An action is \hl{\textbf{\textit{faithful}}} if its kernel is the identity.
\end{enumerate}
\end{defn}

\nl

\begin{cor}
Let $G$ be a group acting on a set $A$. Two elements of $G$ induce the same permutation on $A$ if and only if they are in the same coset.
\end{cor}

\nl

\begin{prop}
Let $G$ be a group acting on the nonempty set $A$. The relation on $A$ defined by
\[a\sim b\quad\text{if and only if}\quad a = g\cdot b\text{ for some } g\in G\]
is an equivalence relation. For each $a\in A$, the number of elements in the equivalence class containing $a$ is $|G:G_a|$, the index of the stabilizer of $a$.
\end{prop}

\nl

\begin{defn}
Let $G$ be a group acting on the nonempty set $A$.
\begin{enumerate}
\item The equivalence class $\{g\cdot a\ |\ g\in G\}$ is called the \textbf{\textit{orbit}} of $G$ containing $a$.
\item The action of $G$ on $A$ is called \hl{\textbf{\textit{transitive}}} if there is only one orbit, i.e., given any two elements $a,b\in A$ there is some $g\in G$ such that $a= g\cdot b$.
\end{enumerate}
\end{defn}

\nl


\begin{thm}
Let $G$ be a group, let $H$ be a subgroup of $G$ and let $G$ act by left multiplication on the set $A$ of left cosets of $H$ in $G$. Let $\pi_H$ be the associated permutation representation afforded by this action. Then
\begin{enumerate}
\item $G$ acts transitively on $A$
\item the stabilizer in $G$ of the point $1H\in A$ is the subgroup $H$
\item the kernel of the action (i.e., the kernel of $\pi_H$) is $\cap_{x\in G}\ xHx\inv$, and $\ker(\pi_H)$ \hl{is the largest normal subgroup of $G$ contained in $H$}.
\end{enumerate}
\end{thm}

\nl

\begin{cor}\hl{\textit{(Cayley's Theorem)}}
Every group is isomorphic to a subgroup of some symmetric group. If $G$ is of order $n$, then $G$ is isomorphic to a subgroup of $S_n$.
\end{cor}

\nl

\begin{cor}
Let $G$ be a simple, non-abelian group and let $H\leq G$. Then $G$ is isomorphic to a subgroup of the symmetric group on $G/H$, $Sym(G/H)$.
\end{cor}

\begin{proof}
Let $G$ be a simple, non-abelian group and let $H\leq G$. Suppose that $G$ acts on the coset space $G/H$ by left multiplication. Obviously, this action is transitive, so we have that there is a homomorphism
\[\vphi:G \ra Sym(G/H): g\mapsto \sig_g\]
where
\[\sig_g: G/H \ra G/H: xH\mapsto (g\cdot x)H.\]
Now, $H$ is a proper subgroup, so $|G/H| > 1$, and since $G$ acts transitively, we have that $\vphi$ is nontrivial. This gives us that $\ker(\vphi) \neq G$, and since $G$ is simple we get that $\vphi$ is injective.
\end{proof}

\nl

\begin{cor}
If $G$ is a finite group of order $n$ and $p$ is the smallest prime dividing $|G|$, then any subgroup of index $p$ is normal. (Note: this is used mostly with subgroups of index 2)
\end{cor}

\nl

\begin{defn}
Two elements $a$ and $b$ of $G$ are said to be \textbf{\textit{conjugate}} in $G$ if there is some $g\in G$ such that $b = gag\inv$. The orbits of $G$ acting on itself by conjugation are called \hl{\textit{conjugacy classes} of $G$.}
\end{defn}

\nl

\begin{defn}
Two subsets $S$ and $T$ of $G$ are said to be \textbf{\textit{conjugate in G}} if there is some $g\in G$ such that $T= gSg\inv$.
\end{defn}

\nl

\begin{prop}
\hl{The number of conjugates of a subset $S$ in a group $G$ is the index of the normalizer of $S$, $|G:N_G(S)|$. In particular, the number of conjugates of an element $s$ of $G$ is the index of the centralizer of $s$, $|G:C_G(s)|$.}
\end{prop}

\nl

\begin{thm}\hl{\textit{(The Class Equation)}}
Let $G$ be a finite group and let $g_1,g_2,\ldots,g_r$ be representatives of the distinct conjugacy classes of $G$ not contained in the center $Z(G)$ of $G$. Then
\[|G| = |Z(G)| + \sum_{i = 1}^r |G:C_G(g_i)|.\]
\end{thm}

\nl

\begin{thm}\hl{\textit{(Orbit Stabilizer Theorem)}}
Let $G$ be a group acting on a set $A$ and consider some $a\in A$. Then
\[|Orb(a)| = |G:Stab(a)|.\]
\end{thm}

\nl

\begin{thm}
\hl{Every normal subgroup is the union of conjugacy classes.}
\end{thm}

\nl 

\begin{defn}
Let $G$ be a group. An isomorphism from $G$ onto itself is called an \textbf{\textit{automorphism}}. The set of all automorphisms of $G$ is denoted by $\Aut(G)$.
\end{defn}

\nl

\begin{prop}
Let $H$ be a normal subgroup of $G$. Then $G$ acts by conjugation on $H$ as automorphisms of $H$. More specifically, the action of $G$ on $H$ by conjugation is defined for each $g\in G$ by 
\[h\mapsto ghg\inv\qquad\text{for each } h\in H.\]
For each $g\in G$, conjugation by $g$ is an automorphism of $H$. The permutation representation afforded by this action is a homomorphism of $G$ into $\Aut(H)$ with kernel $C_G(H)$. In particular, $G/C_G(H)$ is isomorphic to a subgroup of $\Aut(H)$.
\end{prop}

\nl

\begin{cor}
If $K$ is any subgroup of the group $G$ and $g\in G$, then $K\cong gKg\inv$. Conjugate elements and conjugate subgroups have the same order.
\end{cor}

\nl

\begin{cor}
For any subgroup $H$ of a group $G$ the quotient group $N_G(H)/C_G(H)$ is isomorphic to a subgroup of $\Aut(H)$. In particular, $G/Z(G)$ is isomorphic to a subgroup of $\Aut(G)$.
\end{cor}

\nl

\begin{defn}
Let $G$ be a group and let $g\in G$. Conjugation by $g$ is called an \textbf{\textit{inner automorphism}} of $G$ and the subgroup of $\Aut(G)$ consisting of all inner automorphisms is denoted by $\Inn(G)$.
\end{defn}

\nl

\textbf{Note:} For any group G we have that 
\[\Inn(G)\cong G/Z(G).\]
This is really useful when proving that $\Aut(G)$ is nontrivial.

\nl

\begin{defn}
A subgroup $H$ of a group $G$ is called \textbf{\textit{characteristic}} in G, denoted $H$ char $G$, if every automorphism of $G$ maps $H$ to itself, i.e., $\sigma(H) = H$ for all $\sigma\in Aut(G)$.
\end{defn}

\nl

\begin{prop}\textit{(Properties of Characteristic Subgroups)}
\begin{enumerate}
\item characteristic subgroups are normal
\item \hl{if $H$ is the unique subgroup of $G$ of a given order, then $H$ is characteristic in $G$}, and 
\item if $K$ char $H$ and $H\unlhd G$, then $K\unlhd G$.
\end{enumerate}
\end{prop}

\nl

\begin{prop}
The automorphism group of the cyclic group of order $n$ is isomorphic to $(\Z/n\Z)^\times$, and abelian group of order $\vphi(n)$  (where $\vphi$ is Euler's function).
\end{prop}

===============

\textbf{Sylow Theorems}

===============
\nl
\begin{defn}
Let $G$ be a group and let $p$ be a prime.
\begin{enumerate}
\item A group of order $p^\al$ for some $\al\geq 0$ is called a $p$\textbf{\textit{-group}}. Subgroups of $G$ which are $p$-groups are called $p$\textbf{\textit{-subgroups}}.
\item If $G$ is a group of order $p^\al m$, where $p\not |m$, then a subgroup of order $p^\al$ is called a \textbf{\textit{Sylow p-subgroup}} of $G$.
\item The set of Sylow $p$-subgroups of $G$ will be denoted by $Syl_p(G)$ and the number of Sylow $p$-subgroups of $G$ will be denoted by $n_p(G)$.
\end{enumerate}
\end{defn}

\nl

\begin{thm}\hl{\textit{(Sylow's Theorem)}}
Let $G$ be a group of order $p^\al m$, where $p$ is a prime not dividing $m$.
\begin{enumerate}
\item Sylow $p$-subgroups of $G$ exist.
\item If $P$ is a Sylow $p$-subgroup of $G$ and $Q$ is any $p$-subgroup of $G$, then there exists $g\in G$ such that $Q\leq gPg\inv$, i.e., $Q$ is contained in some conjugate of $P$. In particular, any two Sylow $p$-subgroups of $G$ are conjugate in $G$.
\item The number of Sylow $p$-subgroups in $G$ is of the form $1+kp$, i.e., 
\[n_p\equiv 1\mod p.\]
Further, $n_p$ is the index in $G$ of the normalizer $N_G(P)$ for any Sylow $p$-subgroup $P$, \hl{hence $n_p$ divides $m$}.
\end{enumerate}
\end{thm}

\nl

\begin{lem}
Let $P\in Syl_p(G)$. If $Q$ is any $p$-subgroup of $G$, then $Q\cap N_G(P)=  Q\cap P$.
\end{lem}

\nl

\begin{thm}
\hl{A nontrivial $p$-group has a nontrivial center.}
\end{thm}

\begin{proof}
Let $G$ be a nontrivial $p$-group, and $P$ the set of order-$p$ elements of $G$. We have seen that $P$ is nonempty, and indeed that $|P|$ is congruent to $-1 \mod p$. Now consider the action of $G$ on $P$ by conjugation. The stabilizer under this action of any $x$ in $P$ is the centralizer $C(x)$ of $x$, which is the subgroup of $G$ consisting of all elements that commute with $x$. The orbit of $x$ then has size $[G:C(x)]$. But $G$ is a $p$-group, so $[G:C(x)]$ is a power of $p$. Hence $[G:C(x)]$ is either 1 or a multiple of $p$. Since $|P|$ is not a multiple of $p$, it follows that at least one of the orbits is a singleton. Then $C(x)=G$, which is to say that $x$ commutes with every element of $G$. We have thus found a nontrivial element $x$ of the center of $G$.
\end{proof}

\nl

\begin{cor}
Let $P$ be a Sylow $p$-subgroup of $G$. Then the following are equivalent:
\begin{enumerate}
\item $P$ is the unique Sylow $p$-subgroup of $G$, i.e., $n_p = 1$
\item $P$ normal in $G$
\item $P$ is characteristic in $G$
\item All subgroups generated by elements of $p$-power order are $p$-groups, i.e., if $X$ is any subset of $G$ such that $|x|$ is a power of $p$ for all $x\in X$, then $\langle X\rangle$ is a $p$-subgroup.
\end{enumerate}
\end{cor}


%################################################################################

\section{Direct and Semidirect Products and Abelian Groups}

\setcounter{thm}{0}

\begin{prop}
Let $G_1, G_2, \ldots, G_n$ be groups and let $G = G_1\times G_2\times \cdots\times G_n$ be their direct product.
\begin{enumerate}
\item For each fixed $i$ the set of elements of $G$ which have the identity of $G_j$ in the $j^{th}$ position for all $j\neq i$ and arbitrary elements of $G_i$ in position $i$ is a subgroup of $G$ isomorphic to $G_i$:
\[G_i \cong \{(1,1,\ldots,1,g_i,1,\ldots,1)\ |\ g_i\in G_i\}.\]
If we identify $G_i$ with this subgroup, then \hl{$G_i \unlhd G$} and 
\[G/G_i\cong G_1\times\cdots\times G_{i-1}\times G_{i+1}\times \cdots\times G_n.\]
\item for each fixed $i$ define $\pi_i:G\ra G_i$ by
\[\pi_i((g_1,g_2,\ldots,g_n)) = g_i.\]
Then $\pi_i$ is a surjective homomorphism with
\begin{align*}
\ker(\pi_i) &= \{(g_1,\ldots,g_{i-1},1,g_{i+1},\ldots,1)\ |\ g_j\in G_j\text{ for all } j\neq i\}\\
&\cong G_1\times\cdots\times G_{i-1}\times G_{i+1}\times \cdots\times G_n.
\end{align*}
\item Under the identifications in part (1), if $x\in G_i$ and $y\in G_j$ then $xy = yx$.
\end{enumerate}
\end{prop}

\nl

\begin{defn}\nl
\begin{enumerate}
\item A group $G$ is \textit{finitely generated} if there is a finite subset $A$ of $G$ such that $G = \langle A\rangle.$
\item For each $r\in \Z$ with $r\geq 0$, let $\Z^r = \Z\times\Z\times\cdots\times\Z$ be the direct product of $r$ copies of the group $\Z$, where $\Z^0 = 1$. The group $\Z^r$ is called the \textit{free abelian group of rank r}.
\end{enumerate}
\end{defn}

\nl

\begin{thm}\hl{\textit{(Fundamental Theorem of Finitely Generated Abelian Groups)}}
Let $G$ be a finitely generated abelian group. Then
\begin{enumerate}
\item 
\[G\cong \Z^r\times Z_{n_1}\times Z_{n_2}\times\cdots\times Z_{n_s}\]
for some integers $r,n_1,n_2,\ldots,n_s$ satisfying the following conditions:
\begin{enumerate}
\item $r\geq 0$ and $n_j \geq 2$ for all $j$, and
\item $n_{i+1}\ |\ n_i$ for $1\leq i\leq s-1$.
\end{enumerate}
\item the expression in (1) is unique.
\end{enumerate}
\end{thm}

\nl

\begin{defn}
The integer $r$ in the previous theorem is called the \hl{\textit{free rank} or \textit{Betti number}} of $G$ and the integers $n_1, n_2,\ldots,n_s$ are called the \textit{invariant factors} of $G$. The description 
\[G\cong \Z^r\times Z_{n_1}\times Z_{n_2}\times\cdots\times Z_{n_s}\]
is called the \textit{invariant factor decomposition} of $G$.
\end{defn}

\nl

\begin{cor}
If $n$ is the product of distinct primes, then up to isomorphism the only abelian group of order $n$ is the cyclic group of order $n$, \hl{$\Z/n\Z = Z_n$}.
\end{cor}

\nl

\begin{thm}
Let $G$ be an abelian group of order $n > 1$ and let the unique factorization of $n$ distinct prime powers be
\[n = p_1^{\al_1}p_2^{\al_2}\cdots p_k^{\al_k}.\]
Then
\begin{enumerate}
\item $G\cong A_1\times A_2\times\cdots\times A_k$, where $|A_i| = p_i^{\al_i}$
\item for each $A\in \{A_1,A_2,\ldots,A_k\}$ with $|A| = p^\al$,
\[A\cong  Z_{p^{\be_1}}\times Z_{p^{\be_2}}\times\cdots\times Z_{p^{\be_t}}\]
with $\be_1\geq\be_2\geq\cdots\geq\be_t\geq 1$ and $\be_1+\be_2+\cdots+\be_t = \al$
\item the decomposition in (1) and (2) is unique.
\end{enumerate}
\end{thm}

\nl

\begin{defn}
The integers $p^{\be_j}$ described in the preceding theorem are called the \textit{elementary divisors} of $G$. The description of $G$ given in the first two parts of the previous theorem is called the \textit{elementary divisor decomposition} of $G$.
\end{defn}

\nl

\begin{prop}
Let $m,n\in \Z^+$
\begin{enumerate}
\item $Z_m\times Z_n\cong Z_{mn}$ if and only if $\gcd(m,n) = 1$.
\item If $n = p_1^{\al_1}p_2^{\al_2}\cdots p_k^{\al_k}$ then $Z_n \cong  Z_{p_1^{\al_1}}\times Z_{p_2^{\al_2}}\times\cdots\times Z_{p_k^{\al_k}}$
\end{enumerate}
\end{prop}

\nl

\begin{defn}
Let $G$ be a group, let $x,y\in G$ and let $A,B$ be nonempty subsets of $G$.
\begin{enumerate}
\item Define $[x,y] = x\inv y\inv xy$, called the \textit{commutator} of $x$ and $y$.
\item Define $[A,B] = \langle [a,b]\ |\ a\in A,\ b\in B\rangle$, the group generated by commutator of elements from $A$ and $B$.
\item Define $G^\p = \langle [x,y]\ |\ x,y\in G\rangle$, the subgroup of $G$ generated by the commutators of elements from $G$, called the \textit{commutator subgroup} of $G$.
\end{enumerate}
\end{defn}

\nl

\begin{prop}
\hl{Let $G$ be a group, let $x,y\in G$ and let $H\leq G$. Then}
\begin{enumerate}
\item $xy = xy[x,y]$.
\item $H\unlhd G$ if and only if $[H,G]\leq H$.
\item $\sigma([x,y]) = [\sigma(x), \sigma(y)]$ for any $\sigma \in \Aut(G)$, $G^\p$ char $G$, and $G/G^\p$ is abelian.
\item $G/G^\p$ is the largest abelian quotient of $G$ in the sense that if $H\unlhd G$ and $G/H$ is abelian, then $G^\p\leq H$. Conversely, if $G^\p\leq H$ and $H\unlhd G$, then $G/H$ is abelian.
\item If $\vphi:G\ra A$ is any homomorphism of $G$ into an abelian group $A$, then $\vphi$ factors through $G^\p$ i.e. $G^\p\leq\ker(\vphi)$ and the following diagram commutes
\end{enumerate}
\begin{center}
\begin{tikzcd}[column sep = 3em]
G\arrow[r]\arrow[rd, swap, "\vphi"] & G/H\arrow[d]\\
& A
\end{tikzcd}
\end{center}
\end{prop}

\nl

\begin{prop}
Let $H$ and $K$ be subgroup of the group $G$. The number of distinct ways of writing each element of the set $HK$ in the form $hk$, for some $h\in H$ and $k\in K$ is $|H\cap K|$. In particular, if $H\cap K = 1$, the each element of $HK$ can be written uniquely as a product $hk$, for some $h\in H$ and $k\in K$.
\end{prop}

\nl

\begin{thm}\hl{\textit{(Product Recognition)}} Suppose $G$ is a group with subgroups $H$ and $K$ such that
\begin{enumerate}
\item $H$ and $K$ are normal in $G$, and
\item $H\cap K = 1$.
\end{enumerate}
Then $HK\cong H\times K$.
\end{thm}

\nl

\begin{defn}
If $G$ is a group and $H$ and $K$ are normal subgroups of $G$ with $H\cap K = 1$ then we call $HK$ the \textit{internal direct product} of $H$ and $K$. We shall call $H\times K$ the \textit{external direct product} of $H$ and $K$ \hl{(Note: This difference purely determines the notation of the elements of the group as these two are isomorphic by the recognition theorem).}
\end{defn}

\nl

\begin{thm}\label{con. semi}
Let $H$ and $K$ be groups and let $\vphi$ be a homomorphism from $K$ into $\Aut(H)$. Let $\cdot$ denote the (left) action of $K$ on $H$ determined by $\vphi$. Let $G$ be the set of ordered pairs $(h,k)$ with $h\in H$ and $k\in K$ and define the following multiplication on $G$:
\[(h_1,k_1)(h_2,k_2) = (h_1(k_1\cdot h_2), k_1k_2).\]
\begin{enumerate}
\item This multiplication makes $G$ into a group of order $|H||K|$.
\item The sets $\{(h, 1)\ |\ h\in H\}$ and $\{(1,k)\ |\ k\in K\}$ are subgroups of $G$ and the maps $h\mapsto (h,1)$ for $h\in H$ and $k\mapsto (1,k)$ for $k\in K$ are isomorphisms of these subgroups with the groups $H$ and $K$ respectively:
\[H\cong \{(h, 1)\ |\ h\in H\}\quad\text{and}\quad K\cong\{(1,k)\ |\ k\in K\}.\]
\item $\wh H = \{(h, 1)\ |\ h\in H\}\unlhd G$
\item $\wh H\cap \wh K = 1$
\item for all $h\in \wh H$ and $k\in\wh K$, $khk\inv = k\cdot h = \vphi(k)(h)$.

\end{enumerate}
\end{thm}


\nl

\begin{defn}
Let $H$ and $K$ be groups and let $\vphi$ be a homomorphism from $K$ into $\Aut(H)$. The group described in \autoref{con. semi} is called the \textit{semidirect product} of $H$ and $K$ with respect to $\vphi$ and will be denoted $H\rtimes_\vphi K$ (or simply $H\rtimes K$).
\end{defn}

\nl

\begin{prop}
Let $H$ and $K$ be groups and let $\vphi:K\ra \Aut(H)$ be a homomorphism. Then the following are equivalent:
\begin{enumerate}
\item the identity (set) map between $H\rtimes K$ and $H\times K$ is a group homomorphism
\item $\vphi$ is the trivial homomorphism from $K$ into $\Aut(H)$
\item $K\unlhd H\rtimes K$.
\end{enumerate}
\end{prop}

\nl

\begin{thm}
Suppose $G$ is a group with subgroups $H$ and $K$ such that
\begin{enumerate}
\item $H$ and $K$ are normal in $G$, and
\item $H\cap K = 1$.
\end{enumerate}
Let $\vphi:K\ra \Aut(H)$ be the homomorphism defined by mapping $k\in K$ to the automorphism of left conjugation by $k$ on $H$. Then $HK\cong H\times K$. In particular, if $G=HK$ with $H$ and $K$ satisfying (1) and (2), then $G$ is the semidirect product of $H$ and $K$.
\end{thm}

\nl

\begin{defn}
Let $H$ be a subgroup of $G$. A subgroup $K$ is called a \textbf{\textit{compliment}} for $H$ in $G$ if $G = HK$ and $H\cap K = 1$.
\end{defn}

%################################################################################

\section{Futher Topics in Group Theory}
\setcounter{thm}{0}

\begin{defn}
A \textbf{\textit{maximal subgroup}} of a group $G$ is a proper subgroup $M$ of $G$ such that there are no subgroups $H$ of $G$ such that $M<H<G$.
\end{defn}

\nl

\begin{thm}
Let $p$ be a prime and let $P$ be a group of order $p^a$, $a\geq 1$. Then
\begin{enumerate}
\item The center of $P$ is nontrivial.
\item If $H$ is a nontrivial normal subgroup of $P$ then $H$ intersects the center non-trivially. In particular, every subgroup of order $p$ is contained in the center.
\item If $H$ is a normal subgroup of $P$ then $H$ contains a subgroup of order $p^b$ that is normal in $P$ for each divisor $p^b$ of $|H|$. \hl{In particular, $P$ has a normal subgroup of order $p^b$ for every $b\in \{1,2,\ldots,a\}$}.
\item Let $H< P$ then $H<N_P(H)$.
\item Every maximal subgroup of $P$ is of index $p$ and is normal in $P$.
\end{enumerate}
\end{thm}

\nl

\begin{defn}\nl
\begin{enumerate}
\item For any (finite or infinite) group $G$ define the following subgroups inductively
\[Z_0(G) = 1,\qquad Z_1(G)= Z(G)\]
and $Z_{i+1}(G)$ is the subgroup of $G$ containing $Z_i(G)$ such that
\[Z_{i+1}(G)/Z_i(G) = Z(G/Z_i(G))\]
(i.e. $Z_{i+1}(G)$ is the complete preimage in $G$ of the center of $G/Z_i(G)$ under the natural projection). The chain of subgroups
\[Z_0(G)\leq Z_1(G)\leq Z_2(G)\leq\cdots\]
is called the\textbf{ \textit{upper central series of $G$}}.
\item A group $G$ is called \textit{\textbf{nilpotent}} if $Z_c(G) = G$ for some $c\in \Z$. The smallest such $c$ is called the \textit{nilpotence class of G}.
\end{enumerate}
\end{defn}

\nl

\begin{prop}
Let $p$ be a prime and let $P$ be a group of order $p^a$. Then $P$ is nilpotent of nilpotence class at most $a-1$ for $a\geq 2$.
\end{prop}

\begin{proof}
For each $i\geq 0,\ P/Z_i(P)$ is a $p$-group, so if 
\[|P/Z_i(P)| > 1\text{ then } Z(P/Z_i(P)\neq 1\]
by \textcolor{red}{Theorem 6.1 (1)}. Thus if $Z_i(P)\neq P$ then we have that $|Z_{i + 1}(P) \geq p|Z_i(P)|$and so $|Z_{i+1}(P)\geq p^{i+1}$. In particular $|Z_a(P)|\geq p^a$, so $P = Z_a(P)$. The only way $P$ could be of nilpotence class exactly equal to $a$ would be if $|Z_i(P)| = p^i$ for all $i$. In this case, however, $Z_{a-2}$ would have index $p^2$ in $P$, so $P/Z_{a-2}(P)$ would be abelian by \textcolor{red}{Corollary 4.9}. But then $P/Z_{a-1}(P)$ would equal its center and so $Z_{a-1}(P)$ would equal $P$ $\lightning$. This proves that the class of $P$ is $\leq a-1$.
\end{proof}

\nl

\begin{thm}
Let $G$ be a finite group, let $p_1, p_2, \ldots,p_s$ be the distinct primes dividing the order, and let $P_i\in Syl_{p_i}(G),\ 1\leq i\leq s$. Then the following are equivalent:
\begin{enumerate}
\item \hl{$G$ is nilpotent}
\item if $H<G$ then $H<N_G(H)$
\item $P_i\unlhd G$ for $1\leq i\leq s$, \hl{i.e., every Sylow subgroup is normal in $G$}
\item $G\cong P_1\times P_2\times \cdots\times P_s$.
\end{enumerate}
\end{thm}

\nl

\begin{cor}
A finite abelian group is the direct product of its Sylow subgroups (all abelian groups are nilpotent of rank 1).
\end{cor}


\nl

\begin{prop}
If $G$ is a finite group such that for all positive integers $n$ dividing its order, $G$ contains at most $n$ elements $x$ satisfying $x^n = 1$, then $G$ is cyclic.
\end{prop}

\nl

\begin{prop}\textit{(Frattini's Argument)}
Let $G$ be a group, let $H$ be a normal subgroup of $G$, and let $P\in Syl_p(H)$. Then $G=HN_G(P)$ and $|G:H|$ divides $|N_G(P)|$.
\end{prop}

\nl

\begin{prop}
A finite group is nilpotent if and only if every maximal subgroup is normal.
\end{prop}

\nl

\begin{defn}
For any (finite or infinite) group $G$ define the following subgroups inductively:
\[G^0 = G,\qquad G^1 = [G,G],\quad\text{and}\quad G^{i+1} = [G,G^i].\]
The chain of groups 
\[G^0\geq G^1\geq G^2\geq\cdots\]
is called the \textbf{\textit{lower central series of $G$}}.
\end{defn}

\nl

\begin{thm}
A group $G$ is nilpotent if and only if $G^n = 1$ for some $n\geq 0$. More precisely, $G$ is nilpotent of class $c$ if and only if $c$ is the smallest nonnegative integer such that $G^c = 1$. If $G$ is nilpotent of class $c$ then
\[G^{c-1} \leq Z_i(G)\quad\text{for all } i\in \{0,1,\ldots,c\}.\]
\end{thm}

\nl

\begin{defn}
For any group $G$ define the following sequence of subgroups inductively:
\[G^{(0)} = G,\qquad G^{(1)} = [G,G],\quad\text{and}\quad G^{(i+1)} = [G^{(i)},G^{(i)}]\quad\text{for all }i\geq 1.\]
This series of subgroups is called the \textit{derived} or \textit{commutator} series of $G$.
\end{defn}

\nl

\begin{thm}
A group $G$ is solvable if and only if $G^{(n)} = 1$ for some $n\geq 0$.
\end{thm}

\begin{proof}
Assume that $G$ is solvable and so possesses a series 
\[1 = H_0\unlhd H_1\unlhd \cdots\unlhd H_s = G\]
such that each factor $H_{i+1}, H_i$ is abelian. We prove by induction that $G^{(i)}\leq H_{s-i}$. This is true for $i = 0$, so assume that $G^{(i)}\leq H_{s-i}$. Then
\[G^{(i+1)} = [G^{(i)},G^{(i)}] \leq [H_{s-i},H_{s-i}].\]
Since $G$ is solvable, we know that $H_{s-i}/H_{s-i-1}$ is abelian. Moreover, $[H_{s-i},H_{s-i}]$ is the commutator subgroup of $H_{s-1}$, so $H_{s-i}/[H_{s-i},H_{s-i}]$ is the largest abelian quotient of $H_{s-i}$ which gives us that $[H_{s-i},H_{s-i}]\leq H_{s-i-1}$. Thus $G^{(i+1)} [H_{s-i},H_{s-i}]\leq H_{s-i-1}$. Since $H_0 = 1$, we have that $G^{(s)} = 1$.

Conversely, if $G^{(n)} = 1$ for some $n\geq 0$ then if we take $H_i = G^{(n - i)}$ we have $H_i$ is the largest abelian quotient of $H_{i+1}$. Thus the commutator series satisfies the condition for solvability.
\end{proof}

\nl

\begin{prop}
Let $G$ and $K$ be groups, let $H$ be a subgroup of $G$, and let $\vphi:G\ra K$ be a surjective homomorphism.
\begin{enumerate}
\item $H^{(i)}\leq G^{(i)}$ for all $i\geq 0$. In particular, if $G$ is solvable, then so is $H$. 
\item $\vphi(G^{(i)}) = K^{(i)}$. In particular, homomorphic images and quotient groups of solvable groups are solvable.
\item If $N\unlhd G$ and both $N$ and $G/N$ are solvable then so is $G$.
\end{enumerate}
\end{prop}

\nl

\begin{thm}
Let $G$ be a finite group.
\begin{enumerate}
\item (Burnside) If $|G| = p^aq^b$ for some primes $p$ and $q$, then $G$ is solvable.
\item (Phillip Hall) If for every prime $p$ dividing $|G|$ we factor the order of $G$ as $|G| = p^a m$ where $\gcd(p,m) = 1$, and $G$ has a subgroup of order $m$, then $G$ is solvable.
\item (Feit-Thompson) If $|G|$is odd then $G$ is solvable.
\item (Thompson) If for every pair of elements $x,y\in G,$ $\langle x,y\rangle$ is a solvable group, then $G$ is solvable.
\end{enumerate}
\end{thm}

\newpage

\subsection{Free Groups}\nl

The basic idea behind a free group $F(S)$ generated by a set $S$ is that there are no relations satisfied by any of the elements of $S$ (in this sense $S$ can be considered "free" of relations). Now, if we let $S$ be an arbitrary set then a \textit{\textbf{word}} in $S$ is a finite sequence of elements of $S$. We can then define $F(S)$ to simply be the set of all words in $S$. We shall use this idea to carry out a formal construction of $F(S)$ for an arbitrary $S$ below.

One of the important properties that reflects the fact that there are no relations that must be satisfied by members of $S$ is that any \textit{map} from the set $S$ to a group $G$ can be \textit{\textbf{uniquely extended}} to a homomorphism from the group $F(S)$ to $G$. This is called the \textit{\textbf{universal property}} of the free group and is what characterizes the group $F(S)$.

\begin{center}
\begin{tikzcd}[column sep = 2cm, row sep = 0.8cm]
S\arrow[r, "inclusion"] \arrow[dr, swap, "\vphi"] & F(S)\arrow[d, "\Phi"]\\
& G
\end{tikzcd}
\end{center}

Now, the difficulty in the construction of $F(S)$ is the proof that the word concatenation operation is both well defined and associative. If we say that $S$ is given as a set of literals, then we can define a set $S\inv$ such that there is a bijection from the set $S$ to the set $S\inv$ as given by sending $s\in S$ to its corresponding $s\inv \in S\inv$. If we then take some singleton set that is not contained in either $S$ or $S\inv$ and call it $\{1\}$. If we then join these sets we can take any $x\in S\cup S\inv\cup\{1\}$ and declare that $x^1 = x$. This allows us to think of words of $S$ as finite products of members of $S$ and their inverses. A word $s = (s_1,s_2,s_3,\ldots)$ is then said to be \textit{reduced} if 
\begin{enumerate}
\item $s_{i + 1} \neq s_i\inv$ for all $i$ with $s_i\neq 1$
\item if $s_k = 1$ for some $k$, then $s_i = 1$ for all $i\geq k$
\end{enumerate}

The reduced word $(1,1,1,\ldots)$ is called the \textit{empty word} and is denoted by 1. If we let $F(S)$ be the set of reduced words on $S$ then we can embed $S$ into $F(S)$ by 
\[s\mapsto (s,1,1,1,\ldots ).\]
Under this set injection we identify $S$ with its image and henceforth consider $S$ as a subset of $F(S)$. We can then introduce a binary operation on the set $F(S)$ to the tune of word concatenation followed by reduction (this is pretty self-explanatory), and with the introduction of this operation we get our first theorem of this section.

\nl

\begin{thm}
$F(S)$ is a group under the binary operation given above.
\end{thm}

\nl

\begin{thm}
Let $G$ be a group, $S$ a set and $\vphi:S\ra G$ a set map. Then there is a unique group homomorphism $\Phi:F(S)\ra G$ such that the following diagram commutes:
\begin{center}
\begin{tikzcd}[column sep = 2cm, row sep = 0.8cm]
S\arrow[r, "inclusion"] \arrow[dr, swap, "\vphi"] & F(S)\arrow[d, "\Phi"]\\
& G
\end{tikzcd}
\end{center}
\end{thm}

\begin{proof}
If such a map were to exist, then $\Phi$ must satisfy $\Phi(s_1^{\varepsilon_1}s_2^{\varepsilon_2}\cdots s_n^{\varepsilon_n}) = \vphi(s_1)^{\varepsilon_1}\vphi(s_2)^{\varepsilon_2}\cdots \vphi(s_n)^{\varepsilon_n}$ if it is so be a homomorphism (which gives us uniqueness), and the fact that this actually is a homomorphism follows almost directly.
\end{proof}

\nl

\begin{defn}
The group $F(S)$ is called the \textit{free group} on the set $S$. A group $F$ is a \textit{free group} if there is some set $S$ such that $F= F(S)$ -- in this case we call $S$ the set of \textit{free generators} of $F$. The cardinality of $S$ is called the \textit{rank} of the free group.
\end{defn}

\nl

\begin{defn}
Let $S$ be a subset of a group $G$ such that $G = \langle S\rangle$.
\begin{enumerate}
\item A \textit{\textbf{presentation}} for $G$ is a pair $(S,R)$, where $R$ is a set of words in $F(S)$ such that the normal closure of $\langle R\rangle$ in $F(S)$ (the smallest normal subgroup containing $\langle R\rangle$) equals the kernel of the homomorphism $\pi:F(S)\ra G$ (where $\pi$ extends the identity map from $S$ to $S$). The elements of $S$ are called \textit{generators} and those of $R$ are called \textit{relations} of $G$.
\item We say $G$ is \textit{finitely generated} if there is a presentation $(S,R)$ such that $S$ is a finites set and we say $G$ is \textit{finitely presented} if there is a presentation $(S,R)$ with both $S$ and $R$ finite sets.
\end{enumerate}
\end{defn}


%################################################################################

\section{Introduction to Rings}

\setcounter{thm}{0}

\begin{defn}\nl
\begin{enumerate}
\item a ring $R$ is a set together with two binary operations $+$ and $\times$ satisfying the following axioms
\begin{enumerate}
\item $(R, +)$ is an abelian group 
\item $\times$ is associative 
\item the distributive laws hold in $R$
\end{enumerate}
\item The ring $R$ is commutative if $\times$ is commutative
\item The ring $R$ is said to have identity if there is an element $1\in R$.
\end{enumerate}

\end{defn}

\nl

\begin{defn}
A ring with identity $R$ is said to be a \textit{division ring} if very nonzero element has a multiplicative inverse. A commutative division ring is called a \textit{field}.
\end{defn}

\nl

\begin{defn}\nl
\begin{enumerate}
\item A nonzero element $a$ of $R$ is called a \textit{zero divisor} if there is a nonzero element $b\in R$ such that $ab = 0$ or $ba = 0$.
\item Assume that $R$ has identity $1\neq 0$. An element $u$ of $R$ is called a \textit{\textbf{unit}} in $R$ if there is some $v$ in $R$ such that $uv=vu=1$. The set of units is denoted $R^\times$.
\end{enumerate}
\end{defn}

\nl

\begin{defn}
\hl{A commutative ring with identity is called an \textit{\textbf{integral domain}} if it has no zero divisors.}
\end{defn}

\nl

\begin{prop}
Assume that $a,b,$ and $c$ are elements of any ring with $a$ not a zero divisor. If $ab=ac$ then either $a=0$ or $b=c$.
\end{prop}

\nl

\begin{cor}
Any finite integral domain is a field.
\end{cor}

\nl

\begin{defn}
A \textit{subring} of the ring $R$ is \hl{a subgroup of $R$ that is closed under multiplication.}
\end{defn}

\nl

\begin{prop}
Let $R$ be an integral domain and let $p(x), q(x)$ be nonzero elements of $R[x]$. Then
\begin{enumerate}
\item $deg(p(x)q(x)) = deg (p(x)) + deg(q(x))$,
\item the units of $R[x]$ are just the units of $R$,
\item $R[x]$ is an integral domain.
\end{enumerate}
\end{prop}

\nl

\begin{defn}
Let $R$ and $S$ be rings.
\begin{enumerate}
\item A \textit{ring homomorphism} is a map $\vphi: R\ra S$ satisfying
\begin{enumerate}
\item $\vphi(a+b) = \vphi(a)+\vphi(b)$ for all $a,b\in R$, and 
\item $\vphi(ab) = \vphi(a)\vphi(b)$ for all $a,b\in R$
\end{enumerate}
\item The \textit{kernel} of the ring homomorphism $\vphi$ is the set of elements that map to $0_S$.
\item A bijective ring homomorphism is called an isomorphism.
\end{enumerate}
\end{defn}

\nl

\begin{prop}
Let $R$ and $S$ be rings and let $\vphi:R\ra S$ be a homomorphism.
\begin{enumerate}
\item The image of $\vphi$ is a subring of $S$.
\item The kernel of $\vphi$ is a subring of $R$. Furthermore, \hl{if $\al\in\ker(\vphi)$ then $r\al$ and $\al r$ are in $\ker(\vphi)$ for every $r\in R$.}
\end{enumerate}
\end{prop}

\nl

\begin{defn}
Let $R$ be a ring, let $I$ be a subset of $R$ and let $r\in R$.
\begin{enumerate}
\item $rI = \{ra\ |\ a\in I\}$
\item A subset $I$ of $R$ is a \textit{\textbf{left ideal}} of $R$ if 
\begin{enumerate}
\item $I$ is a subring of $R$, and
\item $I$ is closed under left multiplication by elements from $R$, i.e., $rI\seq I$ for all $r\in R$.
\end{enumerate}
There is a similar definition for a right ideal.
\item A subset $I$ that is both a left ideal and a right ideal is called an ideal of $R$.
\end{enumerate}
\end{defn}

\nl

\begin{prop}
Let $R$ be a ring and let $I$ be an ideal of $R$. Then the (additive) quotient group $R/I$ is a ring under the binary operations:
\[(r+I)+(s+I) = (r+s)+I\qquad\text{and}\qquad(r+I)\times(s+I)=(rs)+I\]
for all $r,s\in R$. Conversely if $I$ is any subgroup such that the above operations are well defined, then $I$ is an ideal of $R$.
\end{prop}

\nl

\begin{defn}
When $I$ is an ideal of $R$ the ring $R/I$ with the operations in the previous proposition is called the \textit{\textbf{quotient ring}} of $R$ by $I$.
\end{defn}

\nl

\begin{thm}\nl
\begin{enumerate}
\item \textit{(The First Isomorphism Theorem for Ring)} If $\vphi:R\ra S$ is a homomorphism of rings, then the kernel of $\vphi$ is an ideal of $R$, the image of $\vphi$ is a subring of $S$, and $R/\ker(\vphi)$ is isomorphic as a ring to $\vphi(R)$.
\item If $I$ is any ideal of $R$, then the map 
\[R\ra R/I\qquad\text{defined by}\qquad r\mapsto r+I\]
is a surjective homomorphism with kernel $I$. Thus every ideal is the kernel of a ring homomorphism and vice versa.
\end{enumerate}
\end{thm}

\nl

\begin{thm}\nl
\begin{enumerate}
\item \textit{(The Second Isomorphism Theorem for Rings)} Let $A$ be a subring and let $B$ be an ideal of $R$. Then $A+B = \{a + b\ |\ a\in A, \ b\in B\}$ is a subring of $R$, $A\cap B$ is an ideal of $A$, and $(A+B)/B\cong A/(A\cap B)$.
\item \textit{(The Third Isomorphism Theorem for Rings)} Let $I$ and $J$ be ideals of $R$ with $I\seq J$. Then $J/I$ is an ideal of $R/I$ and $(R/I)/(J/I)\cong R/J$.
\item \textit{(The Fourth or Lattice Isomorphism Theorem for Rings)} Let $I$ be an ideal of $R$. The correspondence $A\leftrightarrow A/I$ is an inclusion preserving bijection between the set of subrings $A$ of $R$ that contain $I$ and the set of subrings of $R/I$ Furthermore, $A$ is an ideal of $R$ if and only if $A/I$ is an ideal of $R/I$.
\end{enumerate}
\end{thm}

\nl

\begin{defn}
Let $R$ be a ring. Then the \hl{\textit{\textbf{characteristic}}} of the ring $R$ is the smallest number $n$ such that $n1 = 1+1+1+\cdots+1 = 0$. If this never happens, then the characteristic of $R$ is said to be $0$.
\end{defn}

\nl

\begin{prop}
\hl{Let $R$ be an integral domain. Then $char(R)$ is either prime or 0.}
\end{prop}

\nl

\begin{defn}
Let $A$ be any subset of the ring $R$.
\begin{enumerate}
\item Let $(A)$ denote the smallest ideal of $R$ containing $A$, called \textbf{\textit{the ideal generated by $A$}}.
\item Let $RA$ denote the set of all finite sums of elements of the form $ra$ with $r\in R$ and $a\in A$.
\item \hl{An ideal generated by a single element is called a \textit{\textbf{principal ideal}}.}
\item An ideal generated by a finite set is called a \textbf{\textit{finitely generated ideal}}.
\end{enumerate}
\end{defn}

\nl

\begin{prop}
Let $I$ be an ideal of $R$.
\begin{enumerate}
\item \hl{$I = R$ if and only if $I$ contains a unit.}
\item Assume $R$ is commutative. Then $R$ is a field if and only if its only ideals are $0$ and $R$.
\end{enumerate}
\end{prop}

\nl

\begin{cor}
If $R$ is a field then any nonzero ring homomorphism from $R$ into another ring is an injection (the kernel of the ring homomorphism is an ideal).
\end{cor}

\nl

\begin{defn}
An ideal $M$ in an arbitrary ring $S$ is called a \textbf{\textit{maximal ideal}} if $M\neq S$ and the only ideals containing $M$ are $M$ and $S$.
\end{defn}

\nl

\begin{prop}
In a ring with identity every proper ideal is contained in a maximal ideal. [NB: This is important because this means ideals in a ring with identity satisfy the ascending chain condition. This becomes really important in the study of infinite rings like the power series ring $\Z\lbb x\rbb$.]
\end{prop}

\nl

\begin{prop}
\hl{Assume $R$ is commutative. The ideal $M$ is maximal if and only if the quotient ring $R/M$ is a field.}
\end{prop}

\nl

\begin{defn}
Assume $R$ is commutative. An ideal $P$ is called a \hl{\textit{\textbf{prime ideal}}} if $P\neq R$ and whenever the product $ab$ of two elements $a,b\in R$ is an element of $P$, then at least one of $a$ and $b$ is an element of $P$.
\end{defn}

\nl

\begin{prop}
\hl{Assume $R$ is commutative. Then the ideal $P$ is a prime ideal in $R$ if and only if the quotient ring $R/P$ is an integral domain.}
\end{prop}

\nl

\begin{cor}
Assume $R$ is commutative. Every maximal ideal of $R$ is a prime ideal.
\end{cor}

\nl

\begin{thm}
Let $R$ be a commutative ring. Let $D$ be any nonempty subset of $R$ that does not contain 0, does not contain any zero divisors, and is closed under multiplication. Then there is a commutative ring $Q$ with 1 such that $Q$ contains $R$ as a subring and every element of $D$ is a unit in $Q$. The ring $Q$ has the following additional properties:
\begin{enumerate}
\item every element of $Q$ is of the form $rd\inv$ for some $r\in R$ and $d\in D$. In particular, if $D = R\backslash\{0\}$ then $Q$ is a field.
\item (uniqueness of $Q$) The ring $Q$ is the \textit{"smallest"} ring containing $R$ in which all the elements of $D$ become units, in the following sense. Let $S$ be any commutative ring with identity and let $\vphi:R\ra S$ be any injective ring homomorphism such that $\vphi(d)$ is a unit in $S$ for every $d\in D$. Then there is an injective homomorphism $\Phi:Q\ra S$ such that $\Phi|_R = \vphi$. In other words, any ring containing an isomorphic copy of $R$ in which all the elements of $D$ become units must also contain an isomorphic copy of $Q$.
\end{enumerate}
\end{thm}

\nl

\begin{defn}
Let $R$, $D$, and $Q$ be as in the above theorem.
\begin{enumerate}
\item The ring $Q$ is called the \textit{\textbf{ring of fractions}} of $D$ with respect to $R$ and is denoted $D\inv R$.
\item If $R$ is an integral domain and $D = R\backslash\{0\}$, $Q$ is called the \textit{\textbf{field of fractions}} or \textit{quotient field} of $R$.
\end{enumerate}
\end{defn}

\nl

\begin{cor}
Let $R$ be an integral domain and let $Q$ be the field of fractions of $R$. If a field $F$ contains a subring $R^\p$ isomorphic to $R$ then the subfield of $F$ generated by $R^\p$ is isomorphic to $Q$.
\end{cor}

\nl

\begin{defn}
The ideals $A$ and $B$ of the ring $R$ are said to be \textbf{\textit{comaximal}} if $A + B = R$.
\end{defn}

\nl

\begin{thm}\hl{\textit{(Chinese Remainder Theorem)}} Let $A_1,A_2,\ldots,A_k$ be ideals in $R$. The map
\[R\ra R/A_1\times R/A_2\times\cdots\times R/A_k\quad\text{defined by}\quad r\mapsto(r+A_1,r+A_2,\ldots,r+A_k)\]
is a ring homomorphism with kernel $\cap A_i$. If for each $i,j\in\{1,2,\ldots,k\}$ with $i\neq j$ the ideals $A_i$ and $A_j$ are comaximal, then this map is surjective and $A_1\cap A_2\cap\cdots\cap A_k = A_1A_2\cdots A_k$, so
\[R/(A_1A_2\cdots A_k) = R/(A_1\cap A_2\cap\cdots\cap A_k) \cong R/A_1\times R/A_2\times\cdots\times R/A_k.\]
\end{thm}

\nl

\begin{cor}
Let $n$ be a positive integer and let $p_1^{\al_1}p_2^{\al_2}\cdots p_k^{\al_k}$ be its factorization into powers of distinct primes. Then
\[Z/n\Z \cong (\Z/p_1^{\al_1}\Z)\times(\Z/p_2^{\al_2}\Z)\times\cdots\times (\Z/p_k^{\al_k}\Z),\]
as rings, so in particular we have the following isomorphism of multiplicative groups:
\[(Z/n\Z)^\times \cong (\Z/p_1^{\al_1}\Z)^\times\times(\Z/p_2^{\al_2}\Z)^\times\times\cdots\times (\Z/p_k^{\al_k}\Z)^\times.\]
\end{cor}

\nl

\begin{cor}
Let $a,b\in \Z$ then
\[\Z/(m)\times \Z/(n) \cong \Z/(\gcd(m,n))\times \Z/(\text{lcm}(m,n))\]
\end{cor}

\begin{proof}(copied from math.stackexchange)
Fix $u,v\in\Bbb Z$ with $un+vm=d$ (Bezout). 
The map $$\Bbb Z_{\operatorname{lcm}(n,m)}\times\Bbb Z_{\gcd(n,m)} \to\Bbb Z_m\times\Bbb Z_n$$
$$ (a+\operatorname{lcm}(n,m)\Bbb Z,b+\gcd(n,m)\Bbb Z)\mapsto(ua+\tfrac mdb+m\Bbb Z,va-\tfrac ndb+n\Bbb Z)$$
is well-defined(!) and clearly a group homomorphism.
For the element on the left to be in the kernel, 
$ua+\tfrac mdb$ must be a multiple of $m$ and $va-\tfrac ndb$ a multiple of $n$.
But then
$$\frac nd\left(ua+\frac mdb\right)+\frac md\left(va-\frac ndb\right) 
=\frac{nu+vm}{d}a=a$$ 
is a multiple of $\frac{nm}d=\operatorname{lcm}(n,m)$, i.e., we may as well assume that $a=0$. Then $\frac mdb$ must be a multiple of $m$, i.e., $b$ a multiple of $d$, i.e. $b\equiv 0$. We conclude that the kernel is trivial and our homomorphism injective. As both groups are finite of same order, the homomoprhism must be an isomorphism.
\end{proof}